\documentclass[11pt]{article}
\usepackage{mathrsfs}
\usepackage{amssymb}
\usepackage{amsmath}
\usepackage{amsthm}
\usepackage{amscd}
\usepackage{epstopdf}
\usepackage{enumerate}
\usepackage{hyperref}
\usepackage{listings}
\usepackage{graphicx}
\usepackage{tikz}
\usepackage{amstext} % for \text macro
\usepackage{array}   % for \newcolumntype macro
\newcolumntype{L}{>{$}l<{$}} % math-mode version of "l" column type
\usepackage{enumitem}
\usepackage[normalem]{ulem}
\allowdisplaybreaks
\graphicspath{{./Images/}}
\hypersetup{
    colorlinks=true,
    linkcolor=blue,
    filecolor=magenta,      
    urlcolor=cyan,
}

\textwidth = 6.5 in
\textheight = 9 in
\oddsidemargin = 0.0 in
\evensidemargin = 0.0 in
\topmargin = 0.0 in
\headheight = 0.0 in
\headsep = 0.0 in
\parskip = 0.2in
\parindent = 0.0in
\theoremstyle{definition}
\newtheorem{theorem}{Theorem}
\newtheorem{conjecture}{Conjecture}
\newtheorem{claim}[theorem]{Claim}
\newtheorem{question}[theorem]{Question}
\newtheorem{problem}[theorem]{Problem}
\newtheorem{lemma}[theorem]{Lemma}
\newtheorem{proposition}[theorem]{Proposition}
\newtheorem{observation}[theorem]{Observation}
\newtheorem{corollary}[theorem]{Corollary}
\newtheorem{Theorem}{Theorem}[section]
\newtheorem{Claim}[Theorem]{Claim}
\newtheorem{Lemma}[Theorem]{Lemma}
\newtheorem{Proposition}[Theorem]{Proposition}
\newtheorem{Corollary}[Theorem]{Corollary}
\newtheorem{definition}[theorem]{Definition}
\newtheorem{example}{Example}
\newtheorem{assumption}{Assumption}
\newtheorem{aside}{Aside}
\newtheorem{fact}{Fact}


\theoremstyle{definition}
\newtheorem{exercise}[theorem]{Exercise}
\newtheorem{remark}[theorem]{Remark}
\newcommand{\N}{\mathbb{N}}
\newcommand{\Q}{\mathbb{Q}}
\newcommand{\Z}{\mathbb{Z}}
\newcommand{\R}{\mathbb{R}}
\newcommand{\C}{\mathbb{C}}
\newcommand{\F}{\mathbb{F}}
\newcommand{\Hom}{\mathrm{Hom}}

\title{Homework 1}
\author{Emma Knight}
\date{Due January 29 at 5PM EST}

\begin{document}
\maketitle
I just want to remind you of a couple of things before I get to the problems.  Firstly, please don't put any identifying information (name, student ID, favorite linux distro, etc.) on your solutions.  Secondly, I am perfectly fine with you to work with each other (in fact, I actively encourage it!), but I do want your submitted write-up to be your own work.  Finally, no late work will be accepted.  If you are having issues with crowdmark, please email me \emph{no later than} 4:45 PM and attach your solutions to the email.  I may not be accessible at exactly that moment but if you do that, then I will make sure that your assignment gets graded.

Onto the mathematics!
\begin{problem}
Let $R = \C[x, y]$.  $R$ is a unique factorization domain but not a PID; this problem is about showing that basically everything in the argument for classification of modules over PIDs goes wrong for modules over $R$.
\begin{enumerate}[label=(\alph*)]
\item Show that there are torsion modules generated by one element over $R$ that are not of the form $R/(f)$ for some element $f \in R$.
\item Show that there are fintie torsion modules over $R$ that are not of the form $\oplus R/I$.
\item Show that there are finite torsion-free modules over $R$ that are not free.
\item Give an example of two finite modules $M, N$ over $R$, with $N$ torsion-free, together with a surjection $M \twoheadrightarrow N$ that doesn't admit a section (i.e. a map $N \rightarrow M$ such that the composition $N \rightarrow M \rightarrow N$ is the identity).
\end{enumerate}
\end{problem}
\begin{problem}This problem is partially review; I think it's still useful to go over again even if you've seen this before in 347.
\begin{enumerate}[label=(\alph*)]
\item Let $R$ be a domain.  Assume that $R$ has a division algorithm: that is, there is a function $d: R\backslash \{0\} \rightarrow \Z_{\geq 0}$ such that, for any two elements $a, b \in R$ there exists $q, r \in R$ such that $a = bq + r$ and either $r = 0$ or $d(r) < d(b)$ (such a ring is called a Euclidean domain).  Show that $R$ is a PID.
\item Let $R$ be a PID.  Show that $R$ is noetherian, and that every irreducible element of $R$ is prime\footnote{An irreducible element $a$ is one such that, if $a = bc$, then exactly one of $b$ or $c$ is a unit in $R$.  A prime element $p$ is an element that generates a prime ideal; this is equivalent to saying that $p|ab$ implies $p|a$ or $p|b$.}.
\item Let $R$ be a noetherian domain.  Show that every element can be written as the product of irreducible elements, and that prime factorizations are unique (up to order and units).  Conclude that PIDs have unique factorization.
\end{enumerate}
\end{problem}
\begin{problem}
Let $k$ be an algebraiclly closed field, $R = k[x]$.  Recall that the irreducible elements of $R$ are just polynomials of the form $x-c$ for $c \in k$.
\begin{enumerate}[label=(\alph*)]
\item Show that $R$ is a Euclidean domain with $d$ being the degree of a polynomial.
\item Give a correspondence between finite torsion $R$-modules and finite-dimensional vector spaces over $k$ with a distinguished endomorphism.
\item Using the classification of finite modules over a PID, prove that every endomorphsim of a finite-dimensional vector space over $k$ can be put in Jordan normal form\footnote{If you are unfamiliar with Jordan normal form, there are many references explaining it; Wikipedia has a good article \href{https://en.wikipedia.org/wiki/Jordan_normal_form}{here}.}.
\end{enumerate}
\end{problem}
\begin{problem}
Do exercise $1.3$ in Eisenbud (if you don't have a copy of Eisenbud an image of this will be posted to learn). This is sometimes stated as ``If $0 \rightarrow M_1 \rightarrow M_2 \rightarrow M_3 \rightarrow 0$ is a short exact sequence of $R$-modules, then $M_2$ is noetherian if and only if $M_1$ and $M_3$ are.''
\end{problem}
\begin{problem}
Let $R = \oplus R_i$ be a graded ring, and $I \subset R$ an ideal.  Define $I_i = R_i \cap I$.  Show that $I$ is homogeneous if and only if $I = \oplus I_i$.  Some people take this as the definition of a homogeneous ideal.
\end{problem}
\begin{problem}
Let $R$ be an arbitrary ring.  An element $a \in R$ is \emph{nilpotent} if and only if there exists some $n >0$ such that $a^n = 0$.
\begin{enumerate}[label=(\alph*)]
\item Show that, if $u$ is a unit and $n$ is nilpotent, then $u+n$ is a unit as well.
\item Show that the set of nilpotents in $R$ form an ideal.
\item Show that, if $n$ is nilpotent and $\mathfrak{p}$ is a prime idea, then $n \in \mathfrak{p}$.
\end{enumerate}
\end{problem}
\begin{problem}
Let $R$ be an arbitrary ring.
\begin{enumerate}[label=(\alph*)]
\item Let $f(x) = a_0 + a_1 x + \cdots + a_n x^n \in R[x]$.  Show that $f$ is a unit if and only if $a_0$ is a unit and $a_i$ is nilpotent for all $i \geq 1$.
\item Keeping notation as above, show that $f$ is nilpotent if and only if all the $a_i$s are.
\item Generalize the previous two items to $R[x_1, \ldots, x_n]$.
\end{enumerate}
\end{problem}
\end{document}