\documentclass[11pt]{article}
\usepackage{mathrsfs}
\usepackage{amssymb}
\usepackage{amsmath}
\usepackage{amsthm}
\usepackage{amscd}
\usepackage{epstopdf}
\usepackage{enumerate}
\usepackage{hyperref}
\usepackage{listings}
\usepackage{graphicx}
\usepackage{tikz}
\usepackage{amstext} % for \text macro
\usepackage{array}   % for \newcolumntype macro
\newcolumntype{L}{>{$}l<{$}} % math-mode version of "l" column type
\usepackage{enumitem}
\usepackage[normalem]{ulem}
\allowdisplaybreaks
\graphicspath{{./Images/}}
\hypersetup{
    colorlinks=true,
    linkcolor=blue,
    filecolor=magenta,      
    urlcolor=cyan,
}

\textwidth = 6.5 in
\textheight = 9 in
\oddsidemargin = 0.0 in
\evensidemargin = 0.0 in
\topmargin = 0.0 in
\headheight = 0.0 in
\headsep = 0.0 in
\parskip = 0.2in
\parindent = 0.0in
\theoremstyle{definition}
\newtheorem{theorem}{Theorem}
\newtheorem{conjecture}{Conjecture}
\newtheorem{claim}[theorem]{Claim}
\newtheorem{question}[theorem]{Question}
\newtheorem{problem}[theorem]{Problem}
\newtheorem{lemma}[theorem]{Lemma}
\newtheorem{proposition}[theorem]{Proposition}
\newtheorem{observation}[theorem]{Observation}
\newtheorem{corollary}[theorem]{Corollary}
\newtheorem{Theorem}{Theorem}[section]
\newtheorem{Claim}[Theorem]{Claim}
\newtheorem{Lemma}[Theorem]{Lemma}
\newtheorem{Proposition}[Theorem]{Proposition}
\newtheorem{Corollary}[Theorem]{Corollary}
\newtheorem{definition}[theorem]{Definition}
\newtheorem{example}{Example}
\newtheorem{assumption}{Assumption}
\newtheorem{aside}{Aside}
\newtheorem{fact}{Fact}


\theoremstyle{definition}
\newtheorem{exercise}[theorem]{Exercise}
\newtheorem{remark}[theorem]{Remark}
\newcommand{\N}{\mathbb{N}}
\newcommand{\Q}{\mathbb{Q}}
\newcommand{\Z}{\mathbb{Z}}
\newcommand{\R}{\mathbb{R}}
\newcommand{\C}{\mathbb{C}}
\newcommand{\F}{\mathbb{F}}
\newcommand{\Hom}{\mathrm{Hom}}

\title{Homework 2}
\author{Emma Knight}
\date{Due Febuary 22 at 5PM EST}

\begin{document}
\maketitle
\begin{problem}This problem is some practice with universal properties.
\begin{enumerate}[label=(\alph*)]
\item Let $M_i$ be modules over a ring.  Recall that $\prod M_i$ admits a map $\pi_n : \prod M_i \rightarrow M_n$ for all $n$.  Show that $\prod M_i$ is universal in the sense that, if $M$ is a module together with a map $f_n: M \rightarrow M_n$ for all $n$, then there exists a unique map $f: M \rightarrow \prod M_i$ such that $f_i = \pi_i \circ f$.
\item Keeping $M_i$ as modules over a ring, show that $\oplus M_i$ admits a similar universal property to the one above but with all arrows reversed.
\item Let $f: M \rightarrow N$ be a map of modules.  Show that $f$ is surjective if and only if for all modules $P$ together with maps $g_1$ and $g_2: N \rightarrow P$ such that $g_1 \circ f = g_2 \circ f$, one must have that $g_1 = g_2$ (a map that satisfies this condition is called an \emph{epimorphism}).  Construct a similar criterion for injectivity (a map satisfying that condition is called a \emph{monomorphism}).
\end{enumerate}
\end{problem}
\begin{problem}Below is some practice with the tensor product.
\begin{enumerate}[label=(\alph*)]
\item Verify that $\otimes$ is right-exact.
\item Given an example of modules over a ring to show that $\otimes$ is not always exact.  That is, give an exact sequence of modules $0\rightarrow M_1\rightarrow M_2\rightarrow M_3\rightarrow 0$ and a module $N$ such that $0\rightarrow M_1\otimes N\rightarrow M_2\otimes N\rightarrow M_3\otimes N\rightarrow 0$ is not exact.
\item Show that $M \otimes R/I \cong M/IM$ (here, $IM$ is the submodule of $M$ generated by elements of the form $im$ with $i \in I$ and $m \in M$).
\item Let $R$ and $S$ be rings, and assume that $S$ is an $R$-algebra (i.e. there is a map $R \rightarrow S$).  Let $M$ be an $R$-module and $N$ be an $S$-module.  Show that $\mathrm{Hom}_{R}(M, N) = \mathrm{Hom}_{S}(M \otimes_{R} S, N)$ as $R$-modules, where in the first $\mathrm{Hom}$ we view $N$ as an $R$-module and in the second one we view it as an $S$-module\footnote{If you're familiar with category theory, then this problem is almost saying that the functor $\mathcal{F}:R-\mathrm{mod} \rightarrow S-\mathrm{mod}$ given by sending $M \rightarrow M\otimes_{R}S$ is adjoint to the functor from $S-\mathrm{mod}$ to $R-\mathrm{mod}$ given by sending $N$ to $N$ (where one forgets about the action of all of $S$ and just uses the action of $R$; all that one needs to show to complete adjointness is naturality of the isomorphism.}.
\end{enumerate}
\end{problem}
\begin{problem}
Do exercise 2.4 in Eisenbud.
\end{problem}
\begin{problem}
Let $k$ be a field of characteristic $0$.  Let $R = k[x]$, $M_1 = R/(x^4-x^2)$, $M_2 = k[x]/x^3 + 1$, $U_1 = \{1, x, x^2, \ldots\}$ and $U_2 = R\backslash (x)$.  Compute $M_i[U_j^{-1}]$ for all $i$ and $j$.
\end{problem}
\begin{problem}
Let $R$ be a PID, and $\mathfrak{p} \subset R$ a nonzero prime ideal, and choose a element $p$ such that $(p) = \mathfrak{p}$.  Let $S = R_{\mathfrak{p}}$ and $K$ be the field of fractions of $R$.  For all $x \in K$, define $v_\mathfrak{p}(x)$ by $v_{\mathfrak{p}}(0) = \infty$ and $v_{\mathfrak{p}}(p^a\frac{x}{y}) = a$ where $a \in \Z$ and $x$ and $y$ are coprime to $p$.  Show that
\begin{enumerate}[label=(\alph*)]
\item For all $x \in K$, at least one of $x$ or $x^{-1}$ is in $S$.
\item $v_{\mathfrak{p}}(xy) = v_{\mathfrak{p}}(x)v_{\mathfrak{p}}(y)$, $v_{\mathfrak{p}}(x+y) \geq \min(v_{\mathfrak{p}}(x), v_{\mathfrak{p}}(y))$, and $v_{\mathfrak{p}}(x+y) = v_{\mathfrak{p}}(x)$ if $v_{\mathfrak{p}}(y) < v_{\mathfrak{p}}(x)$\footnote{Something you can do if you are interested: show that, if $y$ is a real number between $0$ and $1$, then the function defined by $|x| := y^{-v_{\mathfrak{p}}(x)}$ satisfies all the properties you want for an norm on a field (i.e. the distance function $d(x_1, x_2) = |x_1-x_2|$ is a metric on $K$).}.
\item Show that $I_n = \{s\in S|v_{\mathfrak{p}}(s) \geq n\}$ is an ideal for any nonnegative integer $n$.
\item Show that every non-zero ideal of $S$ is of the form $I_n$ for some $n$.
\end{enumerate}
\end{problem}
\begin{problem}
Do exercise 2.9 of Eisenbud.
\end{problem}
\begin{problem}
Do exercise 2.10 of Eisenbud.  Additionally, explain why the ``truly trivial'' statement is, indeed, truly trivial.
\end{problem}
%These are separate problems because it makes life easier for grading
\end{document}