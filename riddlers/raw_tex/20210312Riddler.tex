\documentclass[11pt]{article}
\usepackage{mathrsfs}
\usepackage{amssymb}
\usepackage{amsmath}
\usepackage{amsthm}
\usepackage{amscd}
\usepackage{epstopdf}
\usepackage{enumerate}
\usepackage[normalem]{ulem}
\usepackage{hyperref}
\usepackage{listings}
\usepackage{graphicx}
\usepackage{tikz}
\usepackage{amstext} % for \text macro
\usepackage{array}   % for \newcolumntype macro
\newcolumntype{L}{>{$}l<{$}} % math-mode version of "l" column type
\usepackage[normalem]{ulem}
\allowdisplaybreaks
\graphicspath{{./Images/}}
\hypersetup{
    colorlinks=true,
    linkcolor=blue,
    filecolor=magenta,      
    urlcolor=cyan,
}

\textwidth = 6.5 in
\textheight = 9 in
\oddsidemargin = 0.0 in
\evensidemargin = 0.0 in
\topmargin = 0.0 in
\headheight = 0.0 in
\headsep = 0.0 in
\parskip = 0.2in
\parindent = 0.0in
\newtheorem{theorem}{Theorem}
\newtheorem{conjecture}{Conjecture}
\newtheorem{claim}[theorem]{Claim}
\newtheorem{question}[theorem]{Question}
\newtheorem{problem}[theorem]{Problem}
\newtheorem{lemma}[theorem]{Lemma}
\newtheorem{proposition}[theorem]{Proposition}
\newtheorem{observation}[theorem]{Observation}
\newtheorem{corollary}[theorem]{Corollary}
\newtheorem{Theorem}{Theorem}[section]
\newtheorem{Claim}[Theorem]{Claim}
\newtheorem{Lemma}[Theorem]{Lemma}
\newtheorem{Proposition}[Theorem]{Proposition}
\newtheorem{Corollary}[Theorem]{Corollary}
\newtheorem{definition}[theorem]{Definition}
\newtheorem{example}{Example}
\newtheorem{assumption}{Assumption}
\newtheorem{aside}{Aside}
\newtheorem{fact}{Fact}


\theoremstyle{definition}
\newtheorem{exercise}[theorem]{Exercise}
\newtheorem{remark}[theorem]{Remark}
\newcommand{\N}{\mathbb{N}}
\newcommand{\Q}{\mathbb{Q}}
\newcommand{\Z}{\mathbb{Z}}
\newcommand{\R}{\mathbb{R}}
\newcommand{\C}{\mathbb{C}}
\newcommand{\F}{\mathbb{F}}
\newcommand{\Hom}{\mathrm{Hom}}

\title{FiveThirtyEight's February 19, 2021 Riddler}
\author{Emma Knight}
\date{\today}
\begin{document}
\maketitle
This week's riddler is about baking pie:
\begin{question}
You are baking a pie, and you have a fixed amount of dough to use for crust.  If your pie is a cylinder, what fraction of the crust should you use on the bottom of the pie to maximize the volume of the pie?
\end{question}
Assume that the pie has radius $r$ and height $h$.  Then the total amount of crust you have is $A = 2\pi r^2 + 2\pi rh$ and the total volume is $V = \pi r^2 h$.

\emph{Solution 1}: This solution uses only material taught in calc I.  Looking along the curve $A = c$, one can compute $0 = \frac{dA}{dr} = 4\pi r + 2\pi h + 2\pi r\frac{dh}{dr}$, and gets that $\frac{dh}{dr} = -2-\frac{h}{r}$.  This means that $\frac{dV}{dr} = 2\pi rh + \pi r^2\left(\frac{dh}{dr}\right) = 2\pi rh - 2\pi r^2 - \pi rh = \pi r(h-2r)$.  Setting $\frac{dV}{dr} = 0$, one gets that $r = 0$ or $h = 2r$.  Clearly the first solution is a minimum, and the relevant solution is $h = 2r$, so $A = 6\pi r^2$.  Since the bottom of the pie has area $\pi r^2$, you should set aside one-sixth of the crust for the bottom.

\emph{Solution 2}: This is a multivariable calc solution.  Write $x$ for the area of the bottom, and $y$ for the area of the side.  One has that $A = 2x + y$ and $4\pi V^2 = xy^2$; since maximizing $V$ and maximizing $4\pi V^2$ are the same thing, I will do the latter.

We need to maximize $xy^2$ subject to $2x+y = c$; using Lagrange multipliers, one gets the equations $y^2 = 2\lambda$ and $2xy = \lambda$.  Equating these two equations, we get $y^2 = 4xy$ so either $y = 0$ (clearly not a maximum) or $y = 4x$.  Thus, the area is $6x$, so, as before, we should set aside one-sixth of the crust for the bottom.

All told, this pie looks more like an enclosed canoli than an actual pie.
\end{document}