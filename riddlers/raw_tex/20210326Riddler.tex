\documentclass[11pt]{article}
\usepackage{mathrsfs}
\usepackage{amssymb}
\usepackage{amsmath}
\usepackage{amsthm}
\usepackage{amscd}
\usepackage{epstopdf}
\usepackage{enumerate}
\usepackage[normalem]{ulem}
\usepackage{hyperref}
\usepackage{listings}
\usepackage{graphicx}
\usepackage{tikz}
\usepackage{amstext} % for \text macro
\usepackage{array}   % for \newcolumntype macro
\newcolumntype{L}{>{$}l<{$}} % math-mode version of "l" column type
\usepackage[normalem]{ulem}
\allowdisplaybreaks
\graphicspath{{./Images/}}
\hypersetup{
    colorlinks=true,
    linkcolor=blue,
    filecolor=magenta,      
    urlcolor=cyan,
}

\textwidth = 6.5 in
\textheight = 9 in
\oddsidemargin = 0.0 in
\evensidemargin = 0.0 in
\topmargin = 0.0 in
\headheight = 0.0 in
\headsep = 0.0 in
\parskip = 0.2in
\parindent = 0.0in
\newtheorem{theorem}{Theorem}
\newtheorem{conjecture}{Conjecture}
\newtheorem{claim}[theorem]{Claim}
\newtheorem{question}[theorem]{Question}
\newtheorem{problem}[theorem]{Problem}
\newtheorem{lemma}[theorem]{Lemma}
\newtheorem{proposition}[theorem]{Proposition}
\newtheorem{observation}[theorem]{Observation}
\newtheorem{corollary}[theorem]{Corollary}
\newtheorem{Theorem}{Theorem}[section]
\newtheorem{Claim}[Theorem]{Claim}
\newtheorem{Lemma}[Theorem]{Lemma}
\newtheorem{Proposition}[Theorem]{Proposition}
\newtheorem{Corollary}[Theorem]{Corollary}
\newtheorem{definition}[theorem]{Definition}
\newtheorem{example}{Example}
\newtheorem{assumption}{Assumption}
\newtheorem{aside}{Aside}
\newtheorem{fact}{Fact}


\theoremstyle{definition}
\newtheorem{exercise}[theorem]{Exercise}
\newtheorem{remark}[theorem]{Remark}
\newcommand{\N}{\mathbb{N}}
\newcommand{\Q}{\mathbb{Q}}
\newcommand{\Z}{\mathbb{Z}}
\newcommand{\R}{\mathbb{R}}
\newcommand{\C}{\mathbb{C}}
\newcommand{\F}{\mathbb{F}}
\newcommand{\Hom}{\mathrm{Hom}}

\title{FiveThirtyEight's March 26, 2021 Riddler}
\author{Emma Knight}
\date{\today}
\begin{document}
\maketitle
This week's riddler is about a variant of basketball:
\begin{question}
The rules for men’s basketball in the Riddler Collegiate Athletic Association’s (RCAA) are a little different from those in the NCAA. In the NCAA, when a player is fouled attempting a 3-point shot and misses, they always get three free throw attempts, regardless of how many fouls the opposing team has committed.

But in the RCAA, a player must earn each additional foul shot by making the previous one. In other words, a player can take a second shot if they make the first, and they can take a third shot if they make the second.

Suppose a player on your team has a known shooting profile: Their probability of making the first free throw is $p$, their probability of making the second is $q$, and their probability of making the third is $r$, such that no two of these probabilities are equal. Meanwhile, their expected number of points made for any given three-point foul (which can be computed from $p$, $q$ and $r$) is also known.

What is the greatest number of distinct shooting profiles that are made up of these three different probabilities - $p$, $q$ and $r$, in some order for the three shots — that can result in the same overall expected number of points?
\end{question}
Let $f(x, y, z)$ be the expected number of points scored assuming that the probabilities are $x$, $y$, and $z$ for the first, second, and third shot respectively.  Then one has that $f(x, y, z) = x + xy + xyz$, as the probability that you score the first point is $x$, the probability that you score the second point is $xy$ (you have an $x$ chance to take it and a $y$ chance to make it) and the probability that you score the third point is $xyz$.  Now, since the last term there is the same if you permute $x$, $y$, and $z$, I will also define $g(x, y, z) = x+xy$.

Now let $p$, $q$, and $r$ be fixed distinct probabilities (i.e. real numbers in $[0, 1]$).  We want to see how many possible ways to put $p$, $q$, and $r$ into $g$ give the same value.  Trivially, if $p =0$, then $g(p, q, r) = g(p, r, q) = 0$, so two is always possible\footnote{This isn't the only way to get two; this is just the easiest.  One can also see that $p = .3$, $q = .2$, and $r = .8$ gives $g(p, q, r) = g(q, r, p) = .36$.}.  Now, I want to show the following:
\begin{enumerate}
\item $g(p, q, r) = g(q, p, r)$ is impossible.
\item $g(p, q, r) = g(r, q, p)$ is impossible.
\item If $g(p, q, r) = g(p, r, q)$, then $p = 0$ and no other permutation of $p$, $q$, and $r$ evaluate to $0$ in $g$.
\end{enumerate}

If $g(p, q, r) = g(q, p, r)$, then $p + pq = q + pq$ so $p = q$, a contradiction.  Similarly, if $g(p, q, r) = g(r, q, p)$, then $p(1+q) = r(1+q)$.  Since $q \geq 0$, one has that $p = r$, again impossible.  If $g(p, q, r) = g(p, r, q)$, then $p + pq = p + pr$, so $p = 0$ or $q=r$.  Since the second is impossible, one has that $p = 0$.  Now, for any other permutation, since $g(x, y, z) = x(1+y)$, we are plugging a positive number in for $x$ and a non-negative number in for $y$, we get that $g(x, y, z) \neq 0$ for any other permutation.

Thus, if you have at least three possibilities, then you can't have a swap among the permutations, so you must have $g(p, q, r) = g(q, r, p) = g(r, p, q)$ (after renaming).  Thus, we have $$p + pq = q+ qr = r + rp.$$  Now, one gets
\begin{align*}
qr(1+p) & = q(p+pq) \\
& = pq + pq^2 \\
& \\
qr(1+p) & = (p + pq - q)(1+p) \\
& = p + p^2 + pq + p^2q - q -pq \\
& = p - q + p^2 + p^2q \\
& \\
pq + pq^2 & = p - q + p^2 + p^2 q \\
0 & = p-q + p^2 - pq + p^2q-pq^2 \\
& = (p-q)(1 + p + pq).
\end{align*}

Since $p \neq q$, one has that $1+p + pq = 0$, or $q = -\frac{1+p}{p}$.  But $p$ must necessarily be positive, so that gives $q$ being negative, which is also impossible.  Thus, there is no way to have three possible shooting profiles have the same expected number of points.
\end{document}