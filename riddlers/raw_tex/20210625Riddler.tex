\documentclass[11pt]{article}
\usepackage{mathrsfs}
\usepackage{amssymb}
\usepackage{amsmath}
\usepackage{amsthm}
\usepackage{amscd}
\usepackage{epstopdf}
\usepackage{enumerate}
\usepackage[normalem]{ulem}
\usepackage{hyperref}
\usepackage{listings}
\usepackage{graphicx}
\usepackage{tikz}
\usepackage{amstext} % for \text macro
\usepackage{array}   % for \newcolumntype macro
\newcolumntype{L}{>{$}l<{$}} % math-mode version of "l" column type
\usepackage[normalem]{ulem}
\allowdisplaybreaks
\graphicspath{{./Images/}}
\hypersetup{
    colorlinks=true,
    linkcolor=blue,
    filecolor=magenta,      
    urlcolor=cyan,
}

\textwidth = 6.5 in
\textheight = 9 in
\oddsidemargin = 0.0 in
\evensidemargin = 0.0 in
\topmargin = 0.0 in
\headheight = 0.0 in
\headsep = 0.0 in
\parskip = 0.2in
\parindent = 0.0in
\newtheorem{theorem}{Theorem}
\newtheorem{conjecture}{Conjecture}
\newtheorem{claim}[theorem]{Claim}
\newtheorem{question}[theorem]{Question}
\newtheorem{problem}[theorem]{Problem}
\newtheorem{lemma}[theorem]{Lemma}
\newtheorem{proposition}[theorem]{Proposition}
\newtheorem{observation}[theorem]{Observation}
\newtheorem{corollary}[theorem]{Corollary}
\newtheorem{Theorem}{Theorem}[section]
\newtheorem{Claim}[Theorem]{Claim}
\newtheorem{Lemma}[Theorem]{Lemma}
\newtheorem{Proposition}[Theorem]{Proposition}
\newtheorem{Corollary}[Theorem]{Corollary}
\newtheorem{definition}[theorem]{Definition}
\newtheorem{example}{Example}
\newtheorem{assumption}{Assumption}
\newtheorem{aside}{Aside}
\newtheorem{fact}{Fact}


\theoremstyle{definition}
\newtheorem{exercise}[theorem]{Exercise}
\newtheorem{remark}[theorem]{Remark}
\newcommand{\N}{\mathbb{N}}
\newcommand{\Q}{\mathbb{Q}}
\newcommand{\Z}{\mathbb{Z}}
\newcommand{\R}{\mathbb{R}}
\newcommand{\C}{\mathbb{C}}
\newcommand{\F}{\mathbb{F}}
\newcommand{\Hom}{\mathrm{Hom}}

\title{FiveThirtyEight's June 25, 2021 Riddler}
\author{Emma Knight}
\date{\today}
\begin{document}
\maketitle
This week's riddler, courtesy of Henk Tijms, is about a game of solitare:
\begin{question}
Riddler solitaire is played with $11$ cards: an ace, a two, a three, a four, a five, a six, a seven, an eight, a nine, a ten, and a joker. Each card is worth its face value in points, while the ace counts for $1$ point. To play a game, you shuffle the cards so they are randomly ordered, and then turn them over one by one. You start with 0 points, and as you flip over each card your score increases by that card’s points - as long as the joker hasn’t shown up. The moment the joker appears, the game is over and your score is 0. The key is that you can stop any moment and walk away with a nonzero score.

What strategy maximizes your expected number of points?

\emph{Extra credit}: With an optimal strategy, how many points would you earn on average in a game of Riddler solitaire?
\end{question}

While I don't have a proof, I see no way to do better than the \emph{first-order strategy}: draw a card from the deck if the expected effect on your score is positive.  I call this the first-order strategy because you only think about what's about to happen and make no consideration to what will happen several turns down the line.

To see what this means in practice, assume that the total value of all face cards in the deck is $t$ (which in this case is $55$), the number of cards left in the deck is $d$, and your current score is $s$.  In this world, if you draw the joker, your score decreases by $s$, and if you draw any other card, your score increases by the value of that card.  Summing over all cards in the deck to get the total effect, you get $(t-s)-s = t-2s$, and so the expected effect on your score is to change it by $\frac{t-2s}{d}$.  This is clearly positive if and only if $2s < t$, so the first-order strategy says to draw if you've scored less than half of the maximum score and to stop if you've drawn more.

In particular, in the case in the riddler, you should draw if you have $27$ or fewer points and stop if you have $28$ or more.

One nice thing about formulating the strategy this way is that it lets you handle arbitrary decks with arbitrary amounts of jokers: if the total value of the non-jokers in the deck is $t$, there are $j$ jokers, and your current score is $s$, then you draw if $(j+1)s < t$.

Doing some quick computer simulations (code at the end), playing the game $1000000$ times with this strategy, I got an average score of roughly $15.46$, with the joker being drawn roughly $49.98\%$ of the time.  Increasing the number of cards in the deck, the odds of a joker being drawn hovers at roughly $50\%$.  Additionally, if you have cards with values $1, 2, \ldots, 2n$, the expected value of this strategy seems to be roughly $\frac{n(n+1)}{2}$.

Below is the python code:
\begin{verbatim}import random

##This code plays a single game, with a deck consisting
##of the cards between 1 and n, as well as one joker.
##It takes the strategy of drawing if and only if that
##increases it's score.

def play(n):
    deck = ['J']
    high = 0
    for i in range(n):
        deck.append(i+1)
        high += i+1
    score = 0
    while(2*score < high):
        random.shuffle(deck)
        card = deck.pop()
        if card == 'J':
            return(0)
        score += card
    return(score)

##This simulates playing the game samples times with a
##deck that consists of the cards between 1 and n and one
##joker.  It prints out the average score and the number
##of times a joker is drawn.

def simulate(n, samples):
    totscore = 0
    jokes = 0
    for i in range(samples):
        s = play(n)
        totscore += s
        if s == 0:
            jokes += 1
    print(totscore/samples, jokes/samples)

simulate(10, 1000000)\end{verbatim}
\end{document}