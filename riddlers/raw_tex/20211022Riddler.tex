\documentclass[11pt]{article}
\usepackage{mathrsfs}
\usepackage{amssymb}
\usepackage{amsmath}
\usepackage{amsthm}
\usepackage{amscd}
\usepackage{epstopdf}
\usepackage{enumerate}
\usepackage[normalem]{ulem}
\usepackage{hyperref}
\usepackage{listings}
\usepackage{graphicx}
\usepackage{tikz}
\usepackage{amstext} % for \text macro
\usepackage{array}   % for \newcolumntype macro
\newcolumntype{L}{>{$}l<{$}} % math-mode version of "l" column type
\usepackage[normalem]{ulem}
\allowdisplaybreaks
\graphicspath{{./Images/}}
\hypersetup{
    colorlinks=true,
    linkcolor=blue,
    filecolor=magenta,      
    urlcolor=cyan,
}

\textwidth = 6.5 in
\textheight = 9 in
\oddsidemargin = 0.0 in
\evensidemargin = 0.0 in
\topmargin = 0.0 in
\headheight = 0.0 in
\headsep = 0.0 in
\parskip = 0.2in
\parindent = 0.0in
\newtheorem{theorem}{Theorem}
\newtheorem{conjecture}{Conjecture}
\newtheorem{claim}[theorem]{Claim}
\newtheorem{question}[theorem]{Question}
\newtheorem{problem}[theorem]{Problem}
\newtheorem{lemma}[theorem]{Lemma}
\newtheorem{proposition}[theorem]{Proposition}
\newtheorem{observation}[theorem]{Observation}
\newtheorem{corollary}[theorem]{Corollary}
\newtheorem{Theorem}{Theorem}[section]
\newtheorem{Claim}[Theorem]{Claim}
\newtheorem{Lemma}[Theorem]{Lemma}
\newtheorem{Proposition}[Theorem]{Proposition}
\newtheorem{Corollary}[Theorem]{Corollary}
\newtheorem{definition}[theorem]{Definition}
\newtheorem{example}{Example}
\newtheorem{assumption}{Assumption}
\newtheorem{aside}{Aside}
\newtheorem{fact}{Fact}


\theoremstyle{definition}
\newtheorem{exercise}[theorem]{Exercise}
\newtheorem{remark}[theorem]{Remark}
\newcommand{\N}{\mathbb{N}}
\newcommand{\Q}{\mathbb{Q}}
\newcommand{\Z}{\mathbb{Z}}
\newcommand{\R}{\mathbb{R}}
\newcommand{\C}{\mathbb{C}}
\newcommand{\F}{\mathbb{F}}
\newcommand{\Hom}{\mathrm{Hom}}

\title{FiveThirtyEight's October 22, 2021 Riddler}
\author{Emma Knight}
\date{\today}
\begin{document}
\maketitle
This week's riddler, courtesy of Allen Gu, is a straightforward probability problem:
\begin{question}
Suppose you have an equilateral triangle. You pick three random points, one along each of its three edges, uniformly along the length of each edge - that is, each point along each edge has the same probability of being selected.

With those three randomly selected points, you can form a new triangle inside the original one. What is the probability that the center of the larger triangle also lies inside the smaller one?
\end{question}
A couple of observations are in order: firstly, if the small triangle doesn't contain the center of the big one, then there is at least one edge that is on the ``wrong'' side of the center.  However, if one edge is on the wrong side of the center, then the other two are on the right side, so the probability that the small triangle doesn't contain the center is three times the probability that one particular edge is on the wrong side of the triangle.

You can draw a picture to help, but I will take the equilateral triangle with verticies at $(0, 0)$, $(1, \sqrt{3})$, and $(2, 0)$.  This triangle has center $(1, \frac{\sqrt{3}}{3})$.  Let $(t, t\sqrt{3})$ be a random point on the left side (where $t$ is chosen uniformly at random in $[0, 1]$).  One immediately sees that if $t < \frac{1}{2}$, then the side of the small triangle connecting the left side with the bottom side will never be on the wrong side.  The line connecting the chosen point to the center is given by $(1-t)(y - t\sqrt{3}) = (\frac{\sqrt{3}}{3} - t\sqrt{3})(x-t)$.  We are interested in when $y = 0$, which occurs when $x = \frac{2t}{3t-1}$ (you are invited to suffer through the algebra yourself).  Therefore, if your point on the left side is $(t, t\sqrt{3})$, then the probability that this line is on the wrong side is $1 - \frac{t}{3t-1} = \frac{2}{3} - \frac{1/3}{3t-1}$  Thus, the probability that this line is on the wrong side is the integral of this.  But, as remarked before, we can just multiply this number by three (and get rid of the awful denominators) to get the probability that the triangle doesn't contain the center.  Thus, we compute:
\begin{align*}
\int_{\frac{1}{2}}^1 2 - \frac{1}{3t-1}dt & = 1 - \int_{\frac{1}{2}}^{1} \frac{dt}{3t-1} \\
& = 1 - \left(\frac{\ln(3t-1)}{3}\right|_{\frac{1}{2}}^1 \\
& = 1 - \frac{\ln(2) - \ln(1/2)}{3} \\
& = 1 - \frac{\ln(4)}{3}
\end{align*}

But this was the probability that the small triangle doesn't contain the center, so the probability that the small triangle does contain the center is $\frac{\ln(4)}{3} \approx .462098$.

I was, however, skeptical of this result, and so I decided to code it up.  With $10000000$ simulations, I got an estimated probability of $.4620586$ which is well within what one would expect.  Below is the code:
\begin{verbatim}
import random
import math

##This is the dot product in the plane.
def dot(p, q):
    return(p[0]*q[0] + p[1]*q[1])

##This tests if p1 and p2 are on the same side of
##the line generated by l1 and l2.
def isSameSide(p1, p2, l1, l2):
    t1 = [p1[0]-l1[0], p1[1]-l1[1]]
    t2 = [p2[0]-l1[0], p2[1]-l1[1]]
    n = [l1[1]-l2[1], l2[0]-l1[0]]
    return((dot(t1, n)*dot(t2,n)) > 0)

##This tests if three points that are ri along their
##sides generate a triangle that contains the center
##of the big triangle.
def containsCenter(r1, r2, r3):
    p1 = [r1, 0]
    p2 = [r2/2, r2*math.sqrt(3/4)]
    p3 = [1-r3/2, r3*math.sqrt(3/4)]
    c = [1/2, math.sqrt(1/12)]
    return (isSameSide(c, [1, 0], p1, p2) and
            isSameSide(c, [0, 0], p1, p3) and
            isSameSide(c, [0, 0], p2, p3))

##This is the main loop where I generate count random
##triangles and see if they contain the center.
count = 10000000
successes = 0

for i in range(count):
    if containsCenter(random.random(), random.random(), random.random()):
        successes += 1

print(successes/count)
print(math.log(4)/3)
\end{verbatim}
\end{document}