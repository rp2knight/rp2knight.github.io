\documentclass[11pt]{article}
\usepackage{mathrsfs}
\usepackage{amssymb}
\usepackage{amsmath}
\usepackage{amsthm}
\usepackage{amscd}
\usepackage{epstopdf}
\usepackage{enumerate}
\usepackage[normalem]{ulem}
\usepackage{hyperref}
\usepackage{listings}
\usepackage{graphicx}
\usepackage{tikz}
\usepackage{amstext} % for \text macro
\usepackage{array}   % for \newcolumntype macro
\newcolumntype{L}{>{$}l<{$}} % math-mode version of "l" column type
\usepackage[normalem]{ulem}
\allowdisplaybreaks
\graphicspath{{./Images/}}
\hypersetup{
    colorlinks=true,
    linkcolor=blue,
    filecolor=magenta,      
    urlcolor=cyan,
}

\textwidth = 6.5 in
\textheight = 9 in
\oddsidemargin = 0.0 in
\evensidemargin = 0.0 in
\topmargin = 0.0 in
\headheight = 0.0 in
\headsep = 0.0 in
\parskip = 0.2in
\parindent = 0.0in
\newtheorem{theorem}{Theorem}
\newtheorem{conjecture}{Conjecture}
\newtheorem{claim}[theorem]{Claim}
\newtheorem{question}[theorem]{Question}
\newtheorem{problem}[theorem]{Problem}
\newtheorem{lemma}[theorem]{Lemma}
\newtheorem{proposition}[theorem]{Proposition}
\newtheorem{observation}[theorem]{Observation}
\newtheorem{corollary}[theorem]{Corollary}
\newtheorem{Theorem}{Theorem}[section]
\newtheorem{Claim}[Theorem]{Claim}
\newtheorem{Lemma}[Theorem]{Lemma}
\newtheorem{Proposition}[Theorem]{Proposition}
\newtheorem{Corollary}[Theorem]{Corollary}
\newtheorem{definition}[theorem]{Definition}
\newtheorem{example}{Example}
\newtheorem{assumption}{Assumption}
\newtheorem{aside}{Aside}
\newtheorem{fact}{Fact}


\theoremstyle{definition}
\newtheorem{exercise}[theorem]{Exercise}
\newtheorem{remark}[theorem]{Remark}
\newcommand{\N}{\mathbb{N}}
\newcommand{\Q}{\mathbb{Q}}
\newcommand{\Z}{\mathbb{Z}}
\newcommand{\R}{\mathbb{R}}
\newcommand{\C}{\mathbb{C}}
\newcommand{\F}{\mathbb{F}}
\newcommand{\Hom}{\mathrm{Hom}}

\title{FiveThirtyEight's Janurary 7, 2022 Riddler}
\author{Emma Knight}
\date{\today}
\begin{document}
\maketitle
This week's riddler is a geometry challenge:
\begin{question}
Amare the ant is traveling within Triangle $ABC$. Angle $A$ measures $15$ degrees, and sides $AB$ and $AC$ both have length $1$.

Amare starts at point $B$ and wants to ultimately arrive on side $AC$. However, the queen of his colony has asked him to make several stops along the way. Specifically, his path must:
\begin{itemize}
\item Start at point B.
\item Second, touch a point on side $AC$
\item Third, touch a point back on side $AB$.
\item Finally, touch a point on $AC$.
\end{itemize}
What is the shortest distance Amare can travel to appease the bizzare queen?
\end{question}

There is a very short answer that I will present shortly.  However, I did five 5 pages of calculus before coming up with that solution so I feel obliged to share this with you.

I will solve the problem where Amare goes from $B$ to $AC$ to $AB$ and stops there, as that is a building block to getting the three stop problem.  It is clear that once Amare touches side $AC$, his best path is to travel straight down to the edge $AB$.  Let $p$ be a point on the edge $AC$ (actually, it's better to think of this as a point in $\R^2$ for reasons that will become apparent shortly), and define $d_1$ to be the distance to $B$, and $d_2$ to be the distance to $AB$.  I will view $d_1$ and $d_2$ as functions of $p$.  One has that $\nabla d_1$ is the unit vector in the direction from $B$ to $p$, and $\nabla d_2 = (0, 1)$, the unit vector pointing straight up.

To figure out where on $AC$ Amare should go, let $v$ be a vector parellel to $AC$.  At the correct point, one has that $v\cdot(\nabla d_1 + \nabla d_2) = 0$ (as that is just setting the directional derivative to $0$).  Since $\nabla d_1$ and $\nabla d_2$ are both unit vectors, the only relevant thing is that the angle between $Bp$ and $v$ is equal to the angle between $v$ and $(0, -1)$; a quick calculation shows that this gives the angle between $AB$ and $Bp$ is $60^\circ$.

To see a picture, the optimal path is as follows:

\begin{tikzpicture}[scale=15]
\draw[gray, thick] (1, 0) node[below right]{B} -- (0, 0) node[below left]{A} -- ({sqrt(2+sqrt(3))/2}, {sqrt(2-sqrt(3))/2}) node[above]{C} -- (1, 0);
\draw[blue, thick] (1, 0) -- ({sqrt(3)/2}, {(sqrt(12)-3)/2}) node[above right]{p} -- ({sqrt(3)/2}, 0);
\draw[blue, dashed, thick] ({sqrt(3)/2}, {(sqrt(12)-3)/2}) -- (.8, {sqrt(3)/5});
\draw[gray, thick] ({sqrt(3)/2}, {(sqrt(12) - 3)/2 - .05}) arc (270:120:.05);
\end{tikzpicture}

This image leads one to imagine AC as a mirror.  Drawing a reflected triangle across AC, one instantly sees what's really going on:

\begin{tikzpicture}[scale=15]
\draw[gray, thick] (1, 0) node[below right]{B} -- (0, 0) node[below left]{A} -- ({sqrt(2+sqrt(3))/2}, {sqrt(2-sqrt(3))/2}) node[above right]{C} -- (1, 0);
\draw[gray, thick] (0, 0) -- ({sqrt(3)/2)}, .5) node[above]{B'} --  ({sqrt(2+sqrt(3))/2}, {sqrt(2-sqrt(3))/2});
\draw[blue, thick] (1, 0) -- ({sqrt(3)/2}, {(sqrt(12)-3)/2}) node[above right]{p} -- ({sqrt(3)/2}, 0);
\draw[blue, dashed, thick] ({sqrt(3)/2}, {(sqrt(12)-3)/2}) -- (.75, {sqrt(3)/4});
\end{tikzpicture}

The fastest path for Amare is to just go directly to AB'.

For the three-step version (i.e. what was actually asked for), it's no surprise that you just draw another triangle reflected about AB', and go directly to AC':

\begin{tikzpicture}[scale=15]
\draw[gray, thick] (1, 0) node[below right]{B} -- (0, 0) node[below left]{A} -- ({sqrt(2+sqrt(3))/2}, {sqrt(2-sqrt(3))/2}) node[above right]{C} -- (1, 0);
\draw[gray, thick] (0, 0) -- ({sqrt(3)/2)}, .5) node[above right]{B'} --  ({sqrt(2+sqrt(3))/2}, {sqrt(2-sqrt(3))/2});
\draw[gray, thick] (0, 0) -- ({sqrt(2)/2}, {sqrt(2)/2}) node[above]{C'} -- ({sqrt(3)/2)}, .5);
\draw[blue, thick] (1, 0) -- ({(3 + sqrt(3))/6}, {(3-sqrt(3))/6}) -- ({(sqrt(3)-1}, 0) -- ({(1+sqrt(3))/4}, {(sqrt(3)-1)/4});
\draw[blue, dashed, thick] ({(3 + sqrt(3))/6}, {(3-sqrt(3))/6}) -- (.5, .5);
\draw[blue, dashed, thick] ({(3-sqrt(3))/2},{(sqrt(3)-1)/2}) --  ({(1+sqrt(3))/4}, {(sqrt(3)-1)/4});
\end{tikzpicture}

With this picture, it's immediately apparent that Amare moves $\frac{\sqrt{2}}{2}$ units, as the point of intersection on $AC'$ is $(\frac12, \frac12)$.
\end{document}