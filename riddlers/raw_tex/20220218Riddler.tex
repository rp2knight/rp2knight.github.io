\documentclass[11pt]{article}
\usepackage{mathrsfs}
\usepackage{amssymb}
\usepackage{amsmath}
\usepackage{amsthm}
\usepackage{amscd}
\usepackage{epstopdf}
\usepackage{enumerate}
\usepackage[normalem]{ulem}
\usepackage{hyperref}
\usepackage{listings}
\usepackage{graphicx}
\usepackage{tikz}
\usepackage{amstext} % for \text macro
\usepackage{array}   % for \newcolumntype macro
\newcolumntype{L}{>{$}l<{$}} % math-mode version of "l" column type
\usepackage[normalem]{ulem}
\allowdisplaybreaks
\graphicspath{{./Images/}}
\hypersetup{
    colorlinks=true,
    linkcolor=blue,
    filecolor=magenta,      
    urlcolor=cyan,
}

\textwidth = 6.5 in
\textheight = 9 in
\oddsidemargin = 0.0 in
\evensidemargin = 0.0 in
\topmargin = 0.0 in
\headheight = 0.0 in
\headsep = 0.0 in
\parskip = 0.2in
\parindent = 0.0in
\newtheorem{theorem}{Theorem}
\newtheorem{conjecture}{Conjecture}
\newtheorem{claim}[theorem]{Claim}
\newtheorem{question}[theorem]{Question}
\newtheorem{problem}[theorem]{Problem}
\newtheorem{lemma}[theorem]{Lemma}
\newtheorem{proposition}[theorem]{Proposition}
\newtheorem{observation}[theorem]{Observation}
\newtheorem{corollary}[theorem]{Corollary}
\newtheorem{Theorem}{Theorem}[section]
\newtheorem{Claim}[Theorem]{Claim}
\newtheorem{Lemma}[Theorem]{Lemma}
\newtheorem{Proposition}[Theorem]{Proposition}
\newtheorem{Corollary}[Theorem]{Corollary}
\newtheorem{definition}[theorem]{Definition}
\newtheorem{example}{Example}
\newtheorem{assumption}{Assumption}
\newtheorem{aside}{Aside}
\newtheorem{fact}{Fact}


\theoremstyle{definition}
\newtheorem{exercise}[theorem]{Exercise}
\newtheorem{remark}[theorem]{Remark}
\newcommand{\N}{\mathbb{N}}
\newcommand{\Q}{\mathbb{Q}}
\newcommand{\Z}{\mathbb{Z}}
\newcommand{\R}{\mathbb{R}}
\newcommand{\C}{\mathbb{C}}
\newcommand{\F}{\mathbb{F}}
\newcommand{\Hom}{\mathrm{Hom}}

\title{FiveThirtyEight's February 18, 2022 Riddler}
\author{Emma Knight}
\date{\today}
\begin{document}
\maketitle

This week's riddler, courtesy of Gary Yane, is about forcing a coin to become lucky:
\begin{question}
I have in my possession $1$ million fair coins. Before you ask, these are not legal tender. Among these, I want to find the ``luckiest'' coin.

I first flip all $1$ million coins simultaneously (I'm great at multitasking like that), discarding any coins that come up tails. I flip all the coins that come up heads a second time, and I again discard any of these coins that come up tails. I repeat this process, over and over again. If at any point I am left with one coin, I declare that to be the ``luckiest'' coin.

But getting to one coin is no sure thing. For example, I might find myself with two coins, flip both of them and have both come up tails. Then I would have zero coins, never having had exactly one coin.

What is the probability that I will - at some point - have exactly one ``luckiest'' coin?
\end{question}
Let $p(n)$ be the probability that Gary has a lucky coin, given that he starts with $n$ coins.  We want to compute $p(1000000)$.

One sees from the definition that $p(0) = 0$ and $p(1) = 1$\footnote{Sadly, the obvious continuation of this doesn't work}.  For $n \geq 2$, one has that $p(n) = \displaystyle{\sum_{k = 0}^{n} \binom{n}{k}\frac{f(k)}{2^k}}$, or that $p(n) = \displaystyle{\sum_{k = 0}^{n-1} \binom{n}{k}\frac{f(k)}{2^k-1}}$.  This formula lets one recursively compute $p(n)$, and doing so in python shows that the values seem to stablize at a number that is roughly $.72135$.

I have no further insight into this; I tried to solve this recursion and got absolutely nowhere.  I'll conclude with my code:
\begin{verbatim}
import math

##This computes the values of p(n) up to top, with
##values being the, well, values.

values = [0, 1]
top = 1000

for n in range(2, top):
    s = 0
    for k in range(n):
        s += values[k]*math.comb(n, k)/(2**n-1)
    values.append(s)

print(values)
\end{verbatim}
\end{document}