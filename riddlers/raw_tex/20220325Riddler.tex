\documentclass[11pt]{article}
\usepackage{mathrsfs}
\usepackage{amssymb}
\usepackage{amsmath}
\usepackage{amsthm}
\usepackage{amscd}
\usepackage{epstopdf}
\usepackage{enumerate}
\usepackage[normalem]{ulem}
\usepackage{hyperref}
\usepackage{listings}
\usepackage{graphicx}
\usepackage{tikz}
\usepackage{amstext} % for \text macro
\usepackage{array}   % for \newcolumntype macro
\newcolumntype{L}{>{$}l<{$}} % math-mode version of "l" column type
\usepackage[normalem]{ulem}
\allowdisplaybreaks
\graphicspath{{./Images/}}
\hypersetup{
    colorlinks=true,
    linkcolor=blue,
    filecolor=magenta,      
    urlcolor=cyan,
}

\textwidth = 6.5 in
\textheight = 9 in
\oddsidemargin = 0.0 in
\evensidemargin = 0.0 in
\topmargin = 0.0 in
\headheight = 0.0 in
\headsep = 0.0 in
\parskip = 0.2in
\parindent = 0.0in
\newtheorem{theorem}{Theorem}
\newtheorem{conjecture}{Conjecture}
\newtheorem{claim}[theorem]{Claim}
\newtheorem{question}[theorem]{Question}
\newtheorem{problem}[theorem]{Problem}
\newtheorem{lemma}[theorem]{Lemma}
\newtheorem{proposition}[theorem]{Proposition}
\newtheorem{observation}[theorem]{Observation}
\newtheorem{corollary}[theorem]{Corollary}
\newtheorem{Theorem}{Theorem}[section]
\newtheorem{Claim}[Theorem]{Claim}
\newtheorem{Lemma}[Theorem]{Lemma}
\newtheorem{Proposition}[Theorem]{Proposition}
\newtheorem{Corollary}[Theorem]{Corollary}
\newtheorem{definition}[theorem]{Definition}
\newtheorem{example}{Example}
\newtheorem{assumption}{Assumption}
\newtheorem{aside}{Aside}
\newtheorem{fact}{Fact}


\theoremstyle{definition}
\newtheorem{exercise}[theorem]{Exercise}
\newtheorem{remark}[theorem]{Remark}
\newcommand{\N}{\mathbb{N}}
\newcommand{\Q}{\mathbb{Q}}
\newcommand{\Z}{\mathbb{Z}}
\newcommand{\R}{\mathbb{R}}
\newcommand{\C}{\mathbb{C}}
\newcommand{\F}{\mathbb{F}}
\newcommand{\Hom}{\mathrm{Hom}}

\title{FiveThirtyEight's March 25, 2022 Riddler}
\author{Emma Knight}
\date{\today}
\begin{document}
\maketitle
This week's riddler is an astronomy puzzle:
\begin{question}
The astronomers of Planet Xiddler are back at it. They have developed a new technology for measuring the radius of a planet by analyzing its cross sections.

And so, they launch a satellite to study a newly discovered, spherical planet. The satellite sends back data about three parallel, equally spaced circular cross sections which have radii $A$, $B$ and $C$ megameters, with $0 < A < B < C$. Based on these values, the scientists calculate the radius of the planet is $R$ megameters. To their astonishment, they find that $A$, $B$, $C$ and $R$ are all whole numbers!

What is the smallest possible radius of the newly discovered planet?
\end{question}

The three-dimensionality of this problem isn't actually needed; this can be solved by looking at a cross-section of the planet that is perpendicular to the original three cross sections, and passing through the origin.  Then one has the following picture:

\begin{tikzpicture}
\draw (5, 0) arc (0:180:5);
\draw[thin, lightgray] (-7, 0) -- (7,0);
\draw[thin, lightgray] (0,0) -- (0, 7);
\draw[very thick] (-4, 0) -- (-4, 3) node[left] {$(x-y, A)$};
\draw[very thick] (-3, 0) -- (-3, 4) node[above left] {$(x, B)$};
\draw[very thick] (-2, 0) -- (-2, {sqrt(21)}) node[above left] {$(x+y, C)$};
\draw[dashed] (-2, {sqrt(21)}) -- (0, 0) node[pos=.5, above right] {$R$};
\draw[thick] (-4, 0) to[out=-30, in=-150] node[below] {$y$} (-3, 0);
\draw[thick] (-3, 0) to[out=-20, in=-160] node[below] {$-x$} (0, 0);
\end{tikzpicture}

One then has the following equations:
\begin{align*}
A^2 + (x-y)^2 & = R^2 \\
B^2 + x^2 & = R^2 \\
C^2 + (x+y)^2 & = R^2.
\end{align*}

One gets in order:
\begin{align*}
A^2 + C^2 + 2x^2 + 2y^2 & = 2B^2 + 2x^2 \\
y^2 & = B^2 - \frac{A^2 + C^2}{2} \\
C^2 + x^2 + 2xy + y^2 & = B^2 + x^2 \\
2xy & = \frac{A^2 - C^2}{2} \\
x & = \frac{A^2 - C^2}{4y} \\
R^2 & = B^2 + \frac{(A^2 - C^2)^2}{16B^2 - 8A^2 - 8C^2}.
\end{align*}

In order for $R$ to be an integer, one needs that quantity on the right to be a perfect square of an integer.  Additionally, for this to be a real circle one needs $y^2 > 0$.  At this point, it is not too hard to loop over values of $A$, $B$, and $C$ to find integer values.  Additionally, since $C < R$, once you find one such quadruple, you only need to continue searching until you exhaust all values with $C$ less than that value of $R$.

It turns out that the smallest such quadruple is $(2, 7, 8, 8)$ for $(A, B, C, R)$; this gives $x = -\sqrt{15}$ and $y = \sqrt{15}$ for the extra variables I introduced above.  Here is the corresponding picture:
\begin{tikzpicture}[scale=.9]
\draw (8,0) arc (0:180:8);
\draw[thin, lightgray] (-9, 0) -- (9, 0);
\draw[thin, lightgray] (0,0) -- (0, 9);
\draw[very thick] ({-sqrt(60)}, 0) -- ({-sqrt(60)}, 2);
\draw[thick] ({-sqrt(60)}, 2) to[out=-60, in=60] node[right] {$2$} (-{sqrt(60)}, 0);
\draw[very thick] ({-sqrt(15)}, 0) -- ({-sqrt(15)}, 7);
\draw[thick] ({-sqrt(15)}, 7) to[out=-70, in=70] node[right] {$7$} (-{sqrt(15)}, 0);
\draw[very thick] (0, 0) -- (0, 8);
\draw[thick] (0, 8) to[out=-70, in=70] node[right] {$8$} (0, 0);
\draw[thick] ({-sqrt(60)}, 0) to[out=-20, in=200] node[below] {$\sqrt{15}$} (-{sqrt(15)}, 0);
\draw[thick] ({-sqrt(15)}, 0) to[out=-20, in=200] node[below] {$\sqrt{15}$} (0, 0);
\end{tikzpicture}

Finally, here is the code:
\begin{verbatim}
import itertools
import math

##top is the highest value of C I search for.  solns
##is a list of all the valid triples (A, B, C, R).
##small is the smallest value of R I have found so far.
##best is the triple with the smallest value of R.
##checks is a list of all possible triples (A, B, C)

top = 100
solns = []
small = top*2
best = []
checks = itertools.combinations(range(1, top+1), 3)

##This loop finds all solutions and determines which one
##has the smallest value of R.

for check in checks:
    a = check[0]
    b = check[1]
    c = check[2]
    if(16*(b**2)-8*(a**2)-8*(c**2) > 0):
        if(((a**2 - c**2)**2) % (16*(b**2)-8*(a**2)-8*(c**2)) == 0):
            r = math.sqrt(((a**2 - c**2)**2) // (16*(b**2)-8*(a**2)-8*(c**2)) + b**2)
            if r.is_integer():
                solns.append([a, b, c, r])
                if (r < small):
                    small = r
                    best = [a, b, c, r]

print(solns)
print(best)
\end{verbatim}
\end{document}