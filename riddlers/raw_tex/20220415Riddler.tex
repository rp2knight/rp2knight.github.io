\documentclass[11pt]{article}
\usepackage{mathrsfs}
\usepackage{amssymb}
\usepackage{amsmath}
\usepackage{amsthm}
\usepackage{amscd}
\usepackage{epstopdf}
\usepackage{enumerate}
\usepackage[normalem]{ulem}
\usepackage{hyperref}
\usepackage{listings}
\usepackage{graphicx}
\usepackage{tikz}
\usepackage{amstext} % for \text macro
\usepackage{array}   % for \newcolumntype macro
\newcolumntype{L}{>{$}l<{$}} % math-mode version of "l" column type
\usepackage[normalem]{ulem}
\allowdisplaybreaks
\graphicspath{{./Images/}}
\hypersetup{
    colorlinks=true,
    linkcolor=blue,
    filecolor=magenta,      
    urlcolor=cyan,
}

\textwidth = 6.5 in
\textheight = 9 in
\oddsidemargin = 0.0 in
\evensidemargin = 0.0 in
\topmargin = 0.0 in
\headheight = 0.0 in
\headsep = 0.0 in
\parskip = 0.2in
\parindent = 0.0in
\newtheorem{theorem}{Theorem}
\newtheorem{conjecture}{Conjecture}
\newtheorem{claim}[theorem]{Claim}
\newtheorem{question}[theorem]{Question}
\newtheorem{problem}[theorem]{Problem}
\newtheorem{lemma}[theorem]{Lemma}
\newtheorem{proposition}[theorem]{Proposition}
\newtheorem{observation}[theorem]{Observation}
\newtheorem{corollary}[theorem]{Corollary}
\newtheorem{Theorem}{Theorem}[section]
\newtheorem{Claim}[Theorem]{Claim}
\newtheorem{Lemma}[Theorem]{Lemma}
\newtheorem{Proposition}[Theorem]{Proposition}
\newtheorem{Corollary}[Theorem]{Corollary}
\newtheorem{definition}[theorem]{Definition}
\newtheorem{example}{Example}
\newtheorem{assumption}{Assumption}
\newtheorem{aside}{Aside}
\newtheorem{fact}{Fact}


\theoremstyle{definition}
\newtheorem{exercise}[theorem]{Exercise}
\newtheorem{remark}[theorem]{Remark}
\newcommand{\N}{\mathbb{N}}
\newcommand{\Q}{\mathbb{Q}}
\newcommand{\Z}{\mathbb{Z}}
\newcommand{\R}{\mathbb{R}}
\newcommand{\C}{\mathbb{C}}
\newcommand{\F}{\mathbb{F}}
\newcommand{\Hom}{\mathrm{Hom}}

\title{FiveThirtyEight's April 15, 2022 Riddler}
\author{Emma Knight}
\date{\today}
\begin{document}
\maketitle
This week's riddler,  courtesy of Daniel Larsen,  is about Carmichael numbers:
\begin{question}
Can you find a Carmichael number which has six digits in base $10$,  and is of the form $ABCABC$ with $A$,  $B$,  and$C$ being digits?
\end{question}
Write $x = ABC$ in base 10.   Then,  the question is ``Can you find an integer $x$ such that $100\leq x\leq 999$ and $1001x$ is a Carmichael nummber?''

Since $1001 = 7 \cdot 11\cdot 13$,  one needs that $1001x$ must be congruent to $1 \pmod{60}$,  and so $x \equiv 41 \pmod{60}$.   There are then fifteen numbers to check.   While it is possible to do this by hand,  it's easier to break it into two cases depending on whether $x$ is prime or composite.

If $x$ is prime,  then there is one more condition for $1001x$ to be a Carmichael number: $1001x \equiv 1 \pmod{x-1}$.   This is equivalent to $1001 \equiv 1 \pmod{x-1}$,  or $x-1|1000$.  There is only one such choice of $x$ that is also $41 \pmod{60}$: $x = 101$.   This gives $101101$ as a solution to the problem.

If $x$ is composite,  let $p$ be the smallest prime divisor of $x$ (which can be at most $29$).   By construction,  $p \neq 2, 3,$ or $5$.  Additionally,  $p \neq 7,  11, $ or $13$ because Carmichael numbers are squarefree.   Additionally,  if $p = 23$ (or $29$),  then $1001x-1$ must be divisible by $11$ (or $7$) which is impossible,  as $11|1001$ (or $7|1001$).  Thus,  $p = 17$ or $p = 19$.

If $p = 17$,  then the smallest integer congruent to $41 \pmod{60}$ is $221$.  But that doesn't work,  as $221 = 13\cdot17$.   The second smallest such number is $1411$,  which is too large.   Thus,  $p\neq17$.  If $p = 19$,  then the smallest choice of $x$ is $1121$ which is too large.

Thus,  the only such number is $101101$.
\end{document}