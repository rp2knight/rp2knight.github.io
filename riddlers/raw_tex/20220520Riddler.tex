\documentclass[11pt]{article}
\usepackage{mathrsfs}
\usepackage{amssymb}
\usepackage{amsmath}
\usepackage{amsthm}
\usepackage{amscd}
\usepackage{epstopdf}
\usepackage{enumerate}
\usepackage[normalem]{ulem}
\usepackage{hyperref}
\usepackage{listings}
\usepackage{graphicx}
\usepackage{tikz}
\usepackage{amstext} % for \text macro
\usepackage{array}   % for \newcolumntype macro
\newcolumntype{L}{>{$}l<{$}} % math-mode version of "l" column type
\usepackage[normalem]{ulem}
\allowdisplaybreaks
\graphicspath{{./Images/}}
\hypersetup{
    colorlinks=true,
    linkcolor=blue,
    filecolor=magenta,      
    urlcolor=cyan,
}

\textwidth = 6.5 in
\textheight = 9 in
\oddsidemargin = 0.0 in
\evensidemargin = 0.0 in
\topmargin = 0.0 in
\headheight = 0.0 in
\headsep = 0.0 in
\parskip = 0.2in
\parindent = 0.0in
\newtheorem{theorem}{Theorem}
\newtheorem{conjecture}{Conjecture}
\newtheorem{claim}[theorem]{Claim}
\newtheorem{question}[theorem]{Question}
\newtheorem{problem}[theorem]{Problem}
\newtheorem{lemma}[theorem]{Lemma}
\newtheorem{proposition}[theorem]{Proposition}
\newtheorem{observation}[theorem]{Observation}
\newtheorem{corollary}[theorem]{Corollary}
\newtheorem{Theorem}{Theorem}[section]
\newtheorem{Claim}[Theorem]{Claim}
\newtheorem{Lemma}[Theorem]{Lemma}
\newtheorem{Proposition}[Theorem]{Proposition}
\newtheorem{Corollary}[Theorem]{Corollary}
\newtheorem{definition}[theorem]{Definition}
\newtheorem{example}{Example}
\newtheorem{assumption}{Assumption}
\newtheorem{aside}{Aside}
\newtheorem{fact}{Fact}


\theoremstyle{definition}
\newtheorem{exercise}[theorem]{Exercise}
\newtheorem{remark}[theorem]{Remark}
\newcommand{\N}{\mathbb{N}}
\newcommand{\Q}{\mathbb{Q}}
\newcommand{\Z}{\mathbb{Z}}
\newcommand{\R}{\mathbb{R}}
\newcommand{\C}{\mathbb{C}}
\newcommand{\F}{\mathbb{F}}
\newcommand{\Hom}{\mathrm{Hom}}

\title{FiveThirtyEight's May 20, 2022 Riddler}
\author{Emma Knight}
\date{\today}
\begin{document}
\maketitle
This week's riddler is about black hole photography:
\begin{question}
Assuming the accretion disk of a black hole is equally likely to be in any plane, what is the probability of it being within $10$ degrees of perpendicular to us, thereby resulting in a spectacular image?
\end{question}

To think about the geometry of the situation, the accretion disk is within $10$ degrees of being perpendicular to us if and only if the normal vector is within $10$ degrees of being parellel to us (note that there are two normal vectors, but choosing a vector that is at most $90$ degrees off being parellel is unique up to a set of measure zero).  Since there is a hemisphere of possible normal vectors (see previous parenthetical), one merely needs to compute the area of this ten degree cap, and divide it by $2\pi$ to get the probability.

This cap is easily parameterized by $f(u, v) = (\cos(u)\sin(v), \sin(u)\sin(v), \cos(v))$ for $u \in [0, 2\pi]$ and $v \in [0, \pi/18]$.  Computing the area is now standard: $f_u = (-\sin(u)\sin(v), \cos(u)\sin(v), 0)$ and $f_v = (\cos(u)\cos(v), \sin(u)\cos(v), -\sin(v))$.  Then $||f_u \times f_v|| = \sin(v)$, so the area is $\displaystyle{\int_0^{2\pi}\int_0^{\pi/18}\sin(v)dvdu}$.  This is a very straightforward integral that works out to $2\pi(1-\cos(\pi/18))$.  Thus, the probability of the disk image being this nice is $1-\cos(\pi/18) \approx .01519$.
\end{document}