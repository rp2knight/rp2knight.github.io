\documentclass[11pt]{article}
\usepackage{mathrsfs}
\usepackage{amssymb}
\usepackage{amsmath}
\usepackage{amsthm}
\usepackage{amscd}
\usepackage{epstopdf}
\usepackage{enumerate}
\usepackage[normalem]{ulem}
\usepackage{hyperref}
\usepackage{listings}
\usepackage{graphicx}
\usepackage{tikz}
\usepackage{amstext} % for \text macro
\usepackage{array}   % for \newcolumntype macro
\newcolumntype{L}{>{$}l<{$}} % math-mode version of "l" column type
\usepackage[normalem]{ulem}
\allowdisplaybreaks
\graphicspath{{./Images/}}
\hypersetup{
    colorlinks=true,
    linkcolor=blue,
    filecolor=magenta,      
    urlcolor=cyan,
}

\textwidth = 6.5 in
\textheight = 9 in
\oddsidemargin = 0.0 in
\evensidemargin = 0.0 in
\topmargin = 0.0 in
\headheight = 0.0 in
\headsep = 0.0 in
\parskip = 0.2in
\parindent = 0.0in
\newtheorem{theorem}{Theorem}
\newtheorem{conjecture}{Conjecture}
\newtheorem{claim}[theorem]{Claim}
\newtheorem{question}[theorem]{Question}
\newtheorem{problem}[theorem]{Problem}
\newtheorem{lemma}[theorem]{Lemma}
\newtheorem{proposition}[theorem]{Proposition}
\newtheorem{observation}[theorem]{Observation}
\newtheorem{corollary}[theorem]{Corollary}
\newtheorem{Theorem}{Theorem}[section]
\newtheorem{Claim}[Theorem]{Claim}
\newtheorem{Lemma}[Theorem]{Lemma}
\newtheorem{Proposition}[Theorem]{Proposition}
\newtheorem{Corollary}[Theorem]{Corollary}
\newtheorem{definition}[theorem]{Definition}
\newtheorem{example}{Example}
\newtheorem{assumption}{Assumption}
\newtheorem{aside}{Aside}
\newtheorem{fact}{Fact}


\theoremstyle{definition}
\newtheorem{exercise}[theorem]{Exercise}
\newtheorem{remark}[theorem]{Remark}
\newcommand{\N}{\mathbb{N}}
\newcommand{\Q}{\mathbb{Q}}
\newcommand{\Z}{\mathbb{Z}}
\newcommand{\R}{\mathbb{R}}
\newcommand{\C}{\mathbb{C}}
\newcommand{\F}{\mathbb{F}}
\newcommand{\Hom}{\mathrm{Hom}}

\title{FiveThirtyEight's June 10, 2022 Riddler}
\author{Emma Knight}
\date{\today}
\begin{document}
\maketitle
This week's riddler is about not catching a grasshopper:
\begin{question}
You are trying to catch a grasshopper on a balance beam that is 1 meter long. Every time you try to catch it, it jumps to a random point along the interval between 20 centimeters left of its current position and 20 centimeters right of its current position.

If the grasshopper is within 20 centimeters of one of the edges, it will not jump off the edge. For example, if it is 10 centimeters from the left edge of the beam, then it will randomly jump to anywhere within 30 centimeters of that edge with equal probability (meaning it will be twice as likely to jump right as it is to jump left).

After many, many failed attempts to catch the grasshopper, where is it most likely to be on the beam? Where is it least likely? And what is the ratio between these respective probabilities?
\end{question}

I will view the grasshopper's position on the beam as a real number $x \in [0, 100]$.  Since, in this question, it doesn't matter where the grasshopper starts, I will assume that the grasshopper starts at $50$.  I will define $X_n$ to be the random variable which is the expected location of the grasshopper after $n$ jumps (so $X_0$ is the delta distribution at $50$ and $X_1$ is uniform on $[30, 70]$), and $p_n$ to be the corresponding probability density function.  To compute $X_{n+1}$ from $X_n$ in general, I need some notation.

Let $a(x) = \max(0, x-20)$, $b(x) = \min(x+20, 100)$, and $\ell(x) = b(x)-a(x)$.  The functions $a$ and $b$ represent the minimum and maximum value that the grasshopper can jump to, and $\ell(x)$ is the length of the interval that the grasshopper can jump to; $\ell(x)$ is the piecewise linear function defined by the points $(0, 20)$, $(20, 40)$, $(80, 40)$, and $(100, 20)$.  One can then, by chasing through definitions, see that the probability density function of the pair $(X_n, X_{n+1})$ is $p_{n, n+1}(x, y) = \frac{p_n(x)}{\ell(x)}$ if $a(x) < y < b(x)$ and $0$ otherwise.  Thus, one has that $\displaystyle{p_{n+1}(y) = \int_{a(y)}^{b(y)} \frac{p_n(x)dx}{\ell(x)}}$.

If one writes $p(x)$ to be the limit of the $p_n(x)$s, then one has that $\displaystyle{p(y) = \int_{a(y)}^{b(y)} \frac{p(x)dx}{\ell(x)}}$.  I have no idea how to solve this from first principles.  However, one can always simulate this, and after doing a simulation of $100000$ grasshoppers on the beam taking $1000$ jumps, you get a histogram that looks like this:

\includegraphics{20220610picture.png}

If you squint, it sure looks like $p(x) \propto \ell(x)$.  To test that, notice that $\displaystyle{\int_{a(y)}^{b(y)} \frac{\ell(x)dx}{\ell(x)}} = \int_{a(y)}^{b(y)} dx = \ell(y)$, which is exactly what you want.  Since the integral of $p(x)$ is $1$, it's straightforward to compute that $p(x) = \frac{\ell(x)}{180}$.

Thus, at the end, the grasshopper is equally likely to be anywhere between $20$ and $80$, and is least likely to be at the endpoints.  The probability that the grasshopper is at a fixed point in $[20, 80]$ is twice the probability that the grasshopper is at one of the endpoints.
\end{document}