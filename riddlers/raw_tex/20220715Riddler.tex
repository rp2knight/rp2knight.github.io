\documentclass[11pt]{article}
\usepackage{mathrsfs}
\usepackage{amssymb}
\usepackage{amsmath}
\usepackage{amsthm}
\usepackage{amscd}
\usepackage{epstopdf}
\usepackage{enumerate}
\usepackage[normalem]{ulem}
\usepackage{hyperref}
\usepackage{listings}
\usepackage{graphicx}
\usepackage{tikz}
\usepackage{amstext} % for \text macro
\usepackage{array}   % for \newcolumntype macro
\newcolumntype{L}{>{$}l<{$}} % math-mode version of "l" column type
\usepackage[normalem]{ulem}
\allowdisplaybreaks
\graphicspath{{./Images/}}
\hypersetup{
    colorlinks=true,
    linkcolor=blue,
    filecolor=magenta,      
    urlcolor=cyan,
}

\textwidth = 6.5 in
\textheight = 9 in
\oddsidemargin = 0.0 in
\evensidemargin = 0.0 in
\topmargin = 0.0 in
\headheight = 0.0 in
\headsep = 0.0 in
\parskip = 0.2in
\parindent = 0.0in
\newtheorem{theorem}{Theorem}
\newtheorem{conjecture}{Conjecture}
\newtheorem{claim}[theorem]{Claim}
\newtheorem{question}[theorem]{Question}
\newtheorem{problem}[theorem]{Problem}
\newtheorem{lemma}[theorem]{Lemma}
\newtheorem{proposition}[theorem]{Proposition}
\newtheorem{observation}[theorem]{Observation}
\newtheorem{corollary}[theorem]{Corollary}
\newtheorem{Theorem}{Theorem}[section]
\newtheorem{Claim}[Theorem]{Claim}
\newtheorem{Lemma}[Theorem]{Lemma}
\newtheorem{Proposition}[Theorem]{Proposition}
\newtheorem{Corollary}[Theorem]{Corollary}
\newtheorem{definition}[theorem]{Definition}
\newtheorem{example}{Example}
\newtheorem{assumption}{Assumption}
\newtheorem{aside}{Aside}
\newtheorem{fact}{Fact}


\theoremstyle{definition}
\newtheorem{exercise}[theorem]{Exercise}
\newtheorem{remark}[theorem]{Remark}
\newcommand{\N}{\mathbb{N}}
\newcommand{\Q}{\mathbb{Q}}
\newcommand{\Z}{\mathbb{Z}}
\newcommand{\R}{\mathbb{R}}
\newcommand{\C}{\mathbb{C}}
\newcommand{\F}{\mathbb{F}}
\newcommand{\Hom}{\mathrm{Hom}}

\title{FiveThirtyEight's July 15, 2022 Riddler}
\author{Emma Knight}
\date{\today}
\begin{document}
\maketitle
This week's riddler is about marking a die:
\begin{question}
I have an unlabeled, six-sided die. I make a single mark on the middle of each face that is parallel to one pair of that face’s four edges.

How many unique ways are there for me to mark the die? (Note: If two ways can be rotated so that they appear the same, then they are considered to be the same marking.)

\emph{Extra credit}: Suppose you can also mark a face along one of its two diagonals. Now how many unique ways are there to mark the die?
\end{question}

This is a classic Burnside's Lemma problem.  Let $G$ be the group of rotations of the unit cube, and let $X$ be the set of ways to mark the die not up to rotation (so there are $2^6 = 64$ elements of $X$).  Then the number of orbits in $X$ under the action of $G$ is $\displaystyle{\frac{1}{|G|}\sum_{g \in G}|X^g|}$ where $X^g$ is the set of elements of $X$ that are fixed by $g$\footnote{The proof of this is almost immediate from the orbit stabilizer theorem: $|G|$ times the number of orbits is $\displaystyle{\sum_{O \in G\backslash X} |G| = \sum_{x \in X} |Stab(x)| = \sum_{x \in X, g \in G \atop gx = x} 1 = \sum_{g \in G} |X^g|}$.}.

Then one can just list out what happens for each type of rotation:

\begin{tabular}{c|c|c}
Type of rotation & Number of such rotations & Size of stabilizer \\ \hline \hline 
Identity & $1$ & $64$  \\\hline 
$90$ degree face rotation & $6$ & $0$ \\ \hline
$180$ degree face rotation & $3$ & $16$ \\ \hline
$120$ degree corner rotation & $8$ & $4$ \\ \hline
$180$ degree edge rotation & $6$ & $8$
\end{tabular}

This gives that the number of ways to cast a die is $\frac{64+48 + 32 + 48}{24} = 8$.

For the extra credit, the calculation is similar:

\begin{tabular}{c|c|c}
Type of rotation & Number of such rotations & Size of stabilizer \\ \hline \hline 
Identity & $1$ & $4096$  \\\hline 
$90$ degree face rotation & $6$ & $0$ \\ \hline
$180$ degree face rotation & $3$ & $256$ \\ \hline
$120$ degree corner rotation & $8$ & $16$ \\ \hline
$180$ degree edge rotation & $6$ & $64$
\end{tabular}

This gives the number of such ways to cast a bonus die is $\frac{4096 + 768 + 128 + 384}{24} = 224$.
\end{document}