\documentclass[11pt]{article}
\usepackage{mathrsfs}
\usepackage{amssymb}
\usepackage{amsmath}
\usepackage{amsthm}
\usepackage{amscd}
\usepackage{epstopdf}
\usepackage{enumerate}
\usepackage[normalem]{ulem}
\usepackage{hyperref}
\usepackage{listings}
\usepackage{graphicx}
\usepackage{tikz}
\usepackage{amstext} % for \text macro
\usepackage{array}   % for \newcolumntype macro
\newcolumntype{L}{>{$}l<{$}} % math-mode version of "l" column type
\usepackage[normalem]{ulem}
\allowdisplaybreaks
\graphicspath{{./Images/}}
\hypersetup{
    colorlinks=true,
    linkcolor=blue,
    filecolor=magenta,      
    urlcolor=cyan,
}

\textwidth = 6.5 in
\textheight = 9 in
\oddsidemargin = 0.0 in
\evensidemargin = 0.0 in
\topmargin = 0.0 in
\headheight = 0.0 in
\headsep = 0.0 in
\parskip = 0.2in
\parindent = 0.0in
\newtheorem{theorem}{Theorem}
\newtheorem{conjecture}{Conjecture}
\newtheorem{claim}[theorem]{Claim}
\newtheorem{question}[theorem]{Question}
\newtheorem{problem}[theorem]{Problem}
\newtheorem{lemma}[theorem]{Lemma}
\newtheorem{proposition}[theorem]{Proposition}
\newtheorem{observation}[theorem]{Observation}
\newtheorem{corollary}[theorem]{Corollary}
\newtheorem{Theorem}{Theorem}[section]
\newtheorem{Claim}[Theorem]{Claim}
\newtheorem{Lemma}[Theorem]{Lemma}
\newtheorem{Proposition}[Theorem]{Proposition}
\newtheorem{Corollary}[Theorem]{Corollary}
\newtheorem{definition}[theorem]{Definition}
\newtheorem{example}{Example}
\newtheorem{assumption}{Assumption}
\newtheorem{aside}{Aside}
\newtheorem{fact}{Fact}


\theoremstyle{definition}
\newtheorem{exercise}[theorem]{Exercise}
\newtheorem{remark}[theorem]{Remark}
\newcommand{\N}{\mathbb{N}}
\newcommand{\Q}{\mathbb{Q}}
\newcommand{\Z}{\mathbb{Z}}
\newcommand{\R}{\mathbb{R}}
\newcommand{\C}{\mathbb{C}}
\newcommand{\F}{\mathbb{F}}
\newcommand{\Hom}{\mathrm{Hom}}

\title{FiveThirtyEight's July 22, 2022 Riddler}
\author{Emma Knight}
\date{\today}
\begin{document}
\maketitle
This week's riddler, courtesy of Alec Stein, Jesse Zymet, and Adam Greenberg, is about losing your marbles:
\begin{question}
At the Riddler Marble Shop, there are four enormous bags of marbles for you to acquire. They are labeled ``red,'' ``green,'' ``blue'' and ``assorted.'' Being the purist that you are, you want to select two bags of marbles that are not assorted, and you’d settle for some combination of red, green or blue.

However, noticing your interest in the bags, the shopkeeper alerts you. ``Buyer beware,'' she warns. ``Some jerk switched around the labels on all four bags. Right now, every single bag is incorrectly labeled.'' To give you a chance of properly identifying the bags you would like, she has kindly allowed you to take two - and only two - marbles out of any of the bags, one at a time.

How can you guarantee that neither of the two bags you take is assorted?
\end{question}
The following is one strategy, not guaranteed to be the only one.

First, you take a marble out of the bag labelled ``assorted.,'' and then you take a marble out of a bag that is labelled the color that the first marble you took was.  There are now two possibiliities:
\begin{itemize}
\item If the second marble's color matched the label of the bag it was in, choose any two bags other than that one.
\item If the second marble's color was not the color of the bag it was in, then choose the bag labelled with the second marble's color and the bag labelled ``assorted.''
\end{itemize}

Why this works: without loss of generality, we may assume that the first marble is red (since the ``assorted'' bag is homogeneous, it must only have red marbles in it).  If there is a red marble in the ``red'' bag, then the ``red'' bag must have assorted marbles, and so the other three bags are homogeneous and choosing any two of them works.

If the second marble is, say, green, then either the ``red'' bag has assorted marbles or it has only green marbles.  If the ``red'' bag has assorted marbles, then, for exactly the same reason as above, the ``green'' and ``assorted'' bags must be homogeneous.  If, on the other hand, the ``red'' bag has only green marbles, then the ``green'' bag cannot have assorted marbles (if it did, then the ``blue'' bag must have blue marbles) and so it must also be homogeneous. 
\end{document}