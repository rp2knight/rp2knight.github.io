\documentclass[11pt]{article}
\usepackage{mathrsfs}
\usepackage{amssymb}
\usepackage{amsmath}
\usepackage{amsthm}
\usepackage{amscd}
\usepackage{epstopdf}
\usepackage{enumerate}
\usepackage{hyperref}
\usepackage{listings}
\usepackage{graphicx}
\usepackage{tikz}
\usepackage{amstext} % for \text macro
\usepackage{array}   % for \newcolumntype macro
\newcolumntype{L}{>{$}l<{$}} % math-mode version of "l" column type
\usepackage[normalem]{ulem}
\allowdisplaybreaks
\graphicspath{{./Images/}}
\hypersetup{
    colorlinks=true,
    linkcolor=blue,
    filecolor=magenta,      
    urlcolor=cyan,
}

\textwidth = 6.5 in
\textheight = 9 in
\oddsidemargin = 0.0 in
\evensidemargin = 0.0 in
\topmargin = 0.0 in
\headheight = 0.0 in
\headsep = 0.0 in
\parskip = 0.2in
\parindent = 0.0in
\newtheorem{theorem}{Theorem}
\newtheorem{conjecture}{Conjecture}
\newtheorem{claim}[theorem]{Claim}
\newtheorem{question}[theorem]{Question}
\newtheorem{problem}[theorem]{Problem}
\newtheorem{lemma}[theorem]{Lemma}
\newtheorem{proposition}[theorem]{Proposition}
\newtheorem{observation}[theorem]{Observation}
\newtheorem{corollary}[theorem]{Corollary}
\newtheorem{Theorem}{Theorem}[section]
\newtheorem{Claim}[Theorem]{Claim}
\newtheorem{Lemma}[Theorem]{Lemma}
\newtheorem{Proposition}[Theorem]{Proposition}
\newtheorem{Corollary}[Theorem]{Corollary}
\newtheorem{definition}[theorem]{Definition}
\newtheorem{example}{Example}
\newtheorem{assumption}{Assumption}
\newtheorem{aside}{Aside}
\newtheorem{fact}{Fact}


\theoremstyle{definition}
\newtheorem{exercise}[theorem]{Exercise}
\newtheorem{remark}[theorem]{Remark}
\newcommand{\N}{\mathbb{N}}
\newcommand{\Q}{\mathbb{Q}}
\newcommand{\Z}{\mathbb{Z}}
\newcommand{\R}{\mathbb{R}}
\newcommand{\C}{\mathbb{C}}
\newcommand{\F}{\mathbb{F}}
\newcommand{\Hom}{\mathrm{Hom}}

\title{FiveThirtyEight's August 21, 2020 Riddler}
\author{Emma Knight}
\date{\today}

\begin{document}
\maketitle
This week's riddler, courtesy of Scott Owaga, is about creating a pen for your hamster:
\begin{question}
Quarantined in your apartment, you decide to entertain yourself by building a large pen for your pet hamster. To create the pen, you have several vertical posts, around which you will wrap a sheet of fabric. The sheet is $1$ meter long - meaning the perimeter of your pen can be at most $1$ meter - and weighs $1$ kilogram, while each post weighs $k$ kilograms.

Over the course of a typical day, your hamster gets bored and likes to change rooms in your apartment. That means you want your pen to be lightweight and easy to move between rooms. The total weight of the posts and the fabric you use should not exceed $1$ kilogram.

For example, if $k = 0.2$, then you could make an equilateral triangle with a perimeter of $0.4$ meters (since $0.4$ meters of the sheet would weigh $0.4$ kilograms), or you could make a square with perimeter of $0.2$ meters. However, you couldn’t make a pentagon, since the weight of five posts would already hit the maximum and leave no room for the sheet.

You want to figure out the best shape in order to enclose the largest area possible. What’s the greatest value of $k$ for which you should use four posts rather than three?

\emph{Extra credit}: For which values of k should you use five posts, six posts, seven posts, and so on?
\end{question}
Define $f_n(k)$ to be the area you can enclose (in square metres) if you wish to make an $n$-gon, when the posts weigh $k$ kilos.  One can start computing, and get $f_3(k) = \frac{\sqrt{3}}{4}\left(\frac{1}{3}-k\right)^2$ and $f_4(k) = \left(\frac{1}{4} - k\right)^2$.  Setting these equal, after some algebra, gives $k = \frac{3-2\sqrt[4]{3}}{12-6\sqrt[4]{3}} = .089642\ldots$.

What happens for large $n$ though?  To compute $f_n(k)$, I will draw a picture:

\begin{tikzpicture}[scale=5]
\draw (1.73205/4, 3/4) -- (1.73205/2, 1/2) -- (1.73205/2, -1/2) node[midway, anchor=west] {$s$} -- (1.73205/4, -3/4);
\filldraw (0,0) circle (1pt);
\draw[dashed] (1/2, -1.73205/2) arc (-60:60:1);
\draw (1.73205/2, 0) -- (0, 0) node[midway, anchor=north] {$r$} -- (1.73205/2, 1/2);
\draw (.2, 0) arc (0:30:.2) node[midway, anchor=west] {$\frac{\pi}{n}$};
\end{tikzpicture}

Here, $s$ is the side length of the $n$-gon, and $r$ is the inner radius.  One sees that $s = \frac{1}{n} - k$, $r = \frac{s}{2}\cot\left(\frac{\pi}{n}\right)$, and $f_n(k) = \frac{n}{2}sr = \frac{n}{4}\cot\left(\frac{\pi}{n}\right)\left(\frac{1}{n} - k\right)^2 = \frac{1}{4\pi}\left(\frac{\pi}{n}\cot\left(\frac{\pi}{n}\right)\right)(1-nk)^2$.  Write $g(x) = \left(\frac{\pi}{x}\cot\left(\frac{\pi}{x}\right)\right)^{1/2}$.  Then, writing $f_n(k) = f_{n+1}(k)$, one gets that $g(n)(1-nk) = g(n+1)(1-(n+1)k)$, or $k = \frac{g(n+1) - g(n)}{(n+1)g(n+1) - ng(n)}$.  Thus, for $k >  \frac{g(n+1) - g(n)}{(n+1)g(n+1) - ng(n)}$, an $n$-gon surrounds more area, but for $k <  \frac{g(n+1) - g(n)}{(n+1)g(n+1) - ng(n)}$, an $(n+1)$-gon surrounds more area.

This is an exact formula, but it doesn't really give the behaviour of when you want to switch.  To get an idea how this grows, one needs to get a good approximation of $g(n)$ for $n \gg 0$.  One can use Taylor expansions to get $y\cot(y) = 1 - \frac{y^2}{3} + O(y^4)$, and so $\left(y\cot(y)\right)^{1/2} = 1 - \frac{y^2}{6} + O(y^4)$.  Thus, $g(n) = 1 - \frac{\pi^2}{6n^2} + O(n^{-4})$.  One then gets that the numerator is $g(n+1)-g(n) = \frac{\pi^2}{6}\left(\frac{2n+1}{n^2(n+1)^2}\right) + O(n^{-4}) = \frac{\pi^2}{3}\left(\frac{1}{n(n+1)^2}\right) + O(n^{-4}) = \frac{\pi^2}{3n^3} + O(n^{-4})$.  Meanwhile, the denominator is $1-\frac{6\pi}{n+1} + \frac{6\pi}{n} + O(n^{-3}) = 1 + \frac{6\pi}{n(n+1)} + O(n^{-3}) = 1+O(n^{-2})$, so one has that $ \frac{g(n+1) - g(n)}{(n+1)g(n+1) - ng(n)} = \left(\frac{\pi^2}{3n^3} + O(n^{-4})\right)(1+O(n^{-2})) = \frac{\pi^2}{3n^3} + O(n^{-4})$.
\end{document}