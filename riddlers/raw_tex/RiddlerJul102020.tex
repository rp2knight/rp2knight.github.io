\documentclass[11pt]{article}
\usepackage{mathrsfs}
\usepackage{amssymb}
\usepackage{amsmath}
\usepackage{amsthm}
\usepackage{amscd}
\usepackage{epstopdf}
\usepackage{enumerate}
\usepackage{hyperref}
\usepackage{listings}
\usepackage{graphicx}
\graphicspath{{./Images/}}
\hypersetup{
    colorlinks=true,
    linkcolor=blue,
    filecolor=magenta,      
    urlcolor=cyan,
}

\textwidth = 6.5 in
\textheight = 9 in
\oddsidemargin = 0.0 in
\evensidemargin = 0.0 in
\topmargin = 0.0 in
\headheight = 0.0 in
\headsep = 0.0 in
\parskip = 0.2in
\parindent = 0.0in
\newtheorem{theorem}{Theorem}
\newtheorem{conjecture}{Conjecture}
\newtheorem{claim}[theorem]{Claim}
\newtheorem{question}[theorem]{Question}
\newtheorem{problem}[theorem]{Problem}
\newtheorem{lemma}[theorem]{Lemma}
\newtheorem{proposition}[theorem]{Proposition}
\newtheorem{observation}[theorem]{Observation}
\newtheorem{corollary}[theorem]{Corollary}
\newtheorem{Theorem}{Theorem}[section]
\newtheorem{Claim}[Theorem]{Claim}
\newtheorem{Lemma}[Theorem]{Lemma}
\newtheorem{Proposition}[Theorem]{Proposition}
\newtheorem{Corollary}[Theorem]{Corollary}
\newtheorem{definition}[theorem]{Definition}
\newtheorem{example}{Example}
\newtheorem{assumption}{Assumption}
\newtheorem{aside}{Aside}
\newtheorem{fact}{Fact}


\theoremstyle{definition}
\newtheorem{exercise}[theorem]{Exercise}
\newtheorem{remark}[theorem]{Remark}
\newcommand{\N}{\mathbb{N}}
\newcommand{\Q}{\mathbb{Q}}
\newcommand{\Z}{\mathbb{Z}}
\newcommand{\R}{\mathbb{R}}
\newcommand{\C}{\mathbb{C}}
\newcommand{\F}{\mathbb{F}}
\newcommand{\Hom}{\mathrm{Hom}}

\title{FiveThirtyEight's July 10, 2020 Riddler}
\author{Emma Knight}
\date{\today}
 
\begin{document}
\maketitle
This week's riddler, from Austin Shapiro, is about stacking rings on a child's toy:

\begin{question}
Mira has a toy with five (extra credit: $n$) rings of different diameters and a tapered column. Each ring has a ``correct'' position on the column, from the largest ring that fits snugly at the bottom to the smallest ring that fits snugly at the top.

Each ring she places will slide down to its correct position, if possible. Otherwise, it will rest on what was previously the topmost ring.

How many ways can she stack the rings, so that there is at least one ring on the column?
\end{question}
First off, I will remove the assumption that there is at least one ring on the column; this can be fixed by subtracting $1$ from my answer later.

Define a function $f(n, b)$ as the number of ways to stack rings when there is $n$ units of height left, and $b$ rings that will fit on the bottom (so the riddler is asking about $f(5, 1)-1$).  Then it is easy to see the recurrence relation for $f(n, b)$:
\begin{align*}
f(0, b) & = 1 \\
f(n, b) & = 1 + bf(n, b) + \sum_{i = 0}^{n-2} f(i, b+n-1-i)
\end{align*}

One can compute (either by hand or with a computer) the first few polynomials:
\begin{align*}
f(0, b) & = 1 \\
f(1, b) & = b + 1 \\
f(2, b) & = b^2 + b + 2 \\
f(3, b) & = b^3  + b^2 +  3b  + 4 \\
f(4, b) & = b^4  + b^3 +  4b^2  + 8b  + 9 \\
f(5, b) & =  b^5  + b^4 + 5b^3  + 13b^2  + 23b  + 23 \\
\end{align*}

This might not be easy to see at a glance, but one can rearrange this as follows:
\begin{align*}
f(0, b) & = 1 \\
f(1, b) & = b + 1 \\
f(2, b) & = b^2 + (b + 1) + 1 \\
f(3, b) & = b^3  + (b+1)^2 +  (b+2)  + 1 \\
f(4, b) & = b^4  + (b+1)^3 + (b+2)^2  + (b+3)  + 1 \\
f(5, b) & =  b^5  + (b+1)^4 + (b+2)^3  + (b+3)^2  + (b+4)  + 1 \\
\end{align*}
A clear pattern emerges: $f(n, b) = \displaystyle{\sum_{i = 0}^{n}} (b+i)^{n-i}$.

And now, I suffer through the recursion.  Assume that $f(n, b) = \displaystyle{\sum_{i = 0}^{n}} (b+i)^{n-i}$ for $n < N$.  I will compute $f(N, b)$:
\begin{align*}
f(N, b) & = 1 + bf(N-1, b) + \sum_{i = 0}^{N-2} f(i, b+N-1-i) \\
& = 1 + b\sum_{\ell = 0}^{N-1}(b+\ell)^{N-\ell-1} + \sum_{i = 0}^{N-2} \sum_{j = 0}^{i}(b+N-1+j-i)^{i-j} \\
& = 1 + \sum_{\ell = 0}^{N-1}b(b+\ell)^{N-\ell-1} + \sum_{i = 0}^{N-2} \sum_{k = 0}^{i}(b+N-1-k)^{k} & (k = i-j) \\
& = 1 + \sum_{\ell = 0}^{N-1}b(b+\ell)^{N-\ell-1} + \sum_{k = 0}^{N-2} \sum_{i = k}^{N-2}(b+N-1-k)^{k} \\
& = 1 + \sum_{\ell = 0}^{N-1}b(b+\ell)^{N-\ell-1} + \sum_{k = 0}^{N-2} (N-1-k)(b+N-1-k)^{k} \\
& = 1 + \sum_{\ell = 0}^{N-1}b(b+\ell)^{N-\ell-1} + \sum_{\ell = 1}^{N-1} \ell(b+\ell)^{N-1-\ell} & (\ell = N-1-k)\\
& = 1 + b^{N-\ell} +\sum_{\ell = 1}^{N-1}\left(b(b+\ell)^{N-\ell-1} + \ell(b+\ell)^{N-1-\ell}\right) \\
& = 1 + b^{N-\ell} +\sum_{\ell = 1}^{N-1}(b+\ell)^{N-\ell} \\
& = \sum_{\ell = 0}^{N}(b+\ell)^{N-\ell}.
\end{align*}
Thus, by induction, $f(n, b) = \displaystyle{\sum_{i = 0}^{n}} (b+i)^{n-i}$ for all $n$.  The number that we are looking for is $f(n, 1)-1$, which is just $\displaystyle{\sum_{i = 0}^{n-1}} (1+i)^{n-i} = \displaystyle{\sum_{j = 1}^{n}} j^{n-j+1}$, as the $i = n$ term is $1$.

How big is this?  Define $g(n) = f(n, 1) -1$.  Then I claim that $\displaystyle{\lim_{n\rightarrow\infty}\frac{\ln(g(n))}{n\ln(n)} = 1}.$  Clearly, $g(n) < n\cdot n^n$ so $\ln(g(n)) < (n+1)\ln(n)$.  Moreover, looking at the $\epsilon n$ term, one has that $g(n) > (\epsilon n)^{(1-\epsilon)n}$.  Taking $\ln$ of both sides, one gets that $\ln(g(n)) > (1-\epsilon)n(\ln(n) + \ln(\epsilon)$.  Dividing by $n \ln(n)$ and choosing $n > (\epsilon)^{-\frac{1}{\epsilon}}$, one gets $\frac{\ln(g(n))}{n\ln(n)} > (1-\epsilon)(1 + \frac{\ln(\epsilon)}{\epsilon^{-1}\ln(-\epsilon)}) = (1-\epsilon)^2 > 1-2\epsilon$, so one sees that $\frac{\ln(g(n))}{n\ln(n)}$ is bounded above and below by functions of $n$ that tend to $1$ (admittedly very very \emph{very} slowly for the lower bound; in order to force $\frac{\ln(g(n))}{n\ln(n)} > \frac{9}{10}$, one needs $n > 20^{20}$) showing the growth claimed.
\end{document}