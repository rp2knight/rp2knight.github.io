\documentclass[11pt]{article}
\usepackage{mathrsfs}
\usepackage{amssymb}
\usepackage{amsmath}
\usepackage{amsthm}
\usepackage{amscd}
\usepackage{epstopdf}
\usepackage{enumerate}
\usepackage{hyperref}
\usepackage{listings}
\usepackage{graphicx}
\graphicspath{{C:/Users/Emma/Documents/Math/MiscStuff/images/}}
\hypersetup{
    colorlinks=true,
    linkcolor=blue,
    filecolor=magenta,      
    urlcolor=cyan,
}

\textwidth = 6.5 in
\textheight = 9 in
\oddsidemargin = 0.0 in
\evensidemargin = 0.0 in
\topmargin = 0.0 in
\headheight = 0.0 in
\headsep = 0.0 in
\parskip = 0.2in
\parindent = 0.0in
\newtheorem{theorem}{Theorem}
\newtheorem{conjecture}{Conjecture}
\newtheorem{claim}[theorem]{Claim}
\newtheorem{question}[theorem]{Question}
\newtheorem{problem}[theorem]{Problem}
\newtheorem{lemma}[theorem]{Lemma}
\newtheorem{proposition}[theorem]{Proposition}
\newtheorem{observation}[theorem]{Observation}
\newtheorem{corollary}[theorem]{Corollary}
\newtheorem{Theorem}{Theorem}[section]
\newtheorem{Claim}[Theorem]{Claim}
\newtheorem{Lemma}[Theorem]{Lemma}
\newtheorem{Proposition}[Theorem]{Proposition}
\newtheorem{Corollary}[Theorem]{Corollary}
\newtheorem{definition}[theorem]{Definition}
\newtheorem{example}{Example}
\newtheorem{assumption}{Assumption}
\newtheorem{aside}{Aside}
\newtheorem{fact}{Fact}


\theoremstyle{definition}
\newtheorem{exercise}[theorem]{Exercise}
\newtheorem{remark}[theorem]{Remark}
\newcommand{\N}{\mathbb{N}}
\newcommand{\Q}{\mathbb{Q}}
\newcommand{\Z}{\mathbb{Z}}
\newcommand{\R}{\mathbb{R}}
\newcommand{\C}{\mathbb{C}}
\newcommand{\F}{\mathbb{F}}
\newcommand{\Hom}{\mathrm{Hom}}

\title{FiveThirtyEight's June 5, 2020 Riddler}
\author{Emma Knight}
\date{\today}

\begin{document}
\maketitle

This week's riddler is about making a sign for a protest:
\begin{question}
Some friends have invited you to a protest, and you'll be making a sign with large lettering. You're filling in the sign's letters by drawing horizontal lines with a marker. The marker has a flat circular tip with a radius of 1 centimeter, and you're holding the marker so that it's upright, perpendicular to the sign.

If you draw lines every two centimeters, the shading doesn't look very uniform - each stroke is indeed 2 centimeters wide, but there appear to be gaps between the strokes. Of course, if you drew many, many lines all bunched together, you'd have a rather uniform shading.

But you don't have all day to make this sign. If the lines can't overlap by more than 1 centimeter - half the diameter of the marker tip - what should this overlap be, in order to achieve a shading that's as uniform as possible? And how uniform will this shading be, say, as measured by the standard deviation in relative ink on the sign?
\end{question}
To answer this, define $$f_a(x) = \begin{cases} \sqrt{1-x^2} & 0 \leq x \leq a-1 \\ \sqrt{1-x^2} + \sqrt{1-(x-a)^2} & a-1 \leq x \leq 1 \\\sqrt{1-(x-a)^2} & 1 \leq x \leq a \end{cases}$$
so that $f_a(x)$ describes the amount of ink when you are $x$ cm from the bottom, given that the center of the bottom marker is on $y = 0$ and the center of the next marker is on $y = a$ (technically, this is off by a factor of two.  However, I don't want to keep superfluous factors of two around so I'm going to ignore this until the end).

The average value of $f_a(x)$ on $[0, a]$ is $\frac{\pi}{2a}$ (the total area is the area of two quarter-circles).  Thus, the variance (which is the square of the standard deviation) is $$v(a) = \frac{1}{a} \int_{0}^a \left(f_a(x)-\frac{\pi}{2a}\right)^2dx = \frac{1}{a}\int_{0}^a f_a(x)^2dx - \frac{\pi^2}{4a^3}.$$  Since $f_a$ is symmetric about $\frac{a}{2}$, we get
\begin{align*}
v(a) & = \frac{1}{a} \int_{0}^a f_a(x)^2dx - \frac{\pi^2}{4a^3} \\
& = \frac{2}{a} \int_{0}^{a/2} f_a(x)^2dx - \frac{\pi^2}{4a^3}\\
& = \frac{2}{a} \int_{0}^{a-1} 1-x^2 dx + \frac{2}{a} \int_{a-1}^{a/2}(1-x^2) + (1-(x-a)^2) + 2\sqrt{(1-x^2)(1-(x-a)^2)}dx - \frac{\pi^2}{4a^3}\\
& = \frac{2}{a} \int_{0}^1 1-x^2 dx + \frac{2}{a} \int_{a-1}^{a/2}2\sqrt{(1-x^2)(1-(x-a)^2)}dx - \frac{\pi^2}{4a^3}\\
& = \frac{4}{3a} + \frac{2}{a} \int_{a-1}^{a/2}2\sqrt{(1-x^2)(1-(x-a)^2)}dx - \frac{\pi^2}{4a^3}\\
& = \frac{4}{3a} + \frac{1}{a} \int_{a-1}^{1}2\sqrt{(1-x^2)(1-(x-a)^2)}dx - \frac{\pi^2}{4a^3}.
\end{align*}
The integral in the last line is an elliptic integral.  This doesn't preclude an exact answer, but it isn't a positive sign.  Differentiating, one gets:
\begin{align*}
v'(a) & = -\frac{8}{3a^2} - \frac{1}{a^2}\int_{a-1}^{1}2\sqrt{(1-x^2)(1-(x-a)^2)}dx + \frac{1}{a}\int_{a-1}^{1}2(x-a)\sqrt{\frac{1-x^2}{1-(x-a)^2}}dx + \frac{3\pi^2}{4a^4} \\
& = -\frac{8}{3a^2} + \frac{1}{a^2}\int_{a-1}^{1}-2\sqrt{(1-x^2)(1-(x-a)^2)} + 2a(x-a)\sqrt{\frac{1-x^2}{1-(x-a)^2}}dx + \frac{3\pi^2}{4a^4} \\
& = -\frac{8}{3a^2} + \frac{1}{a^2}\int_{a-1}^{1}(-2(1-(x-a)^2) + 2a(x-a))\sqrt{\frac{1-x^2}{1-(x-a)^2}}dx + \frac{3\pi^2}{4a^4} \\
& = -\frac{8}{3a^2} + \frac{1}{a^2}\int_{a-1}^{1}2(x^2-xa-1)\sqrt{\frac{1-x^2}{1-(x-a)^2}}dx + \frac{3\pi^2}{4a^4} \\
\end{align*}
This integral is, in fact, still an elliptic integral.  There is no way to evaluate this integral in a closed form, and so it is impossible to continue on from here with an exact answer.  Thus, I will turn to the wonderful world of code.

\begin{verbatim}
import math
import scipy.integrate as integrate
import matplotlib.pyplot as plt

##This is the function f_a(x) in the writeup.  This is the
##density of ink at one spot given a gap of a.
def f(x, a):
    if ((x >= -1) and (x < a-1)):
        return(math.sqrt(1-x**2))
    if ((x >= a-1) and (x < 1)):
        return(math.sqrt(1-x**2)+math.sqrt(1-(x-a)**2))
    if ((x >= 1) and (x <= a+1)):
        return(math.sqrt(1-(x-a)**2))
    else:
        return(0)

##This gives the average amount of ink and the variance
##in amount of ink for a given value of a.
def avg(a):
    return(math.pi/(2*a))

def variance(a):
    integral = integrate.quad(lambda x: (f(x, a)-avg(a))**2, 0, a/2)
    return(integral[0]*2/a)

##These are some basic variables that are used.  accuracy
##is the inverse of the accuracy for the calculation.  xs
##and ys are for graphing things.  lowvalue is the lowest
##value of the variance, and lowpoint is the a value where
##lowvalue is attained.
accuracy = 100000
xs = []
ys = []
lowvalue = 1000
lowpoint = 0

##This is the main loop.  This just graphs the variance, and
##computes the lowest value of it.
for i in range(accuracy+1):
    x = (i/accuracy)+1
    y = variance(x)
    if (abs(y) < lowvalue):
        lowvalue = abs(y)
        lowpoint = x
    xs.append(x)
    ys.append(y)

plt.plot(xs, ys, 'r-')
plt.show()

print(lowpoint, lowvalue)

plt.close()

##This is for displaying the graph of ink density for the lowest
##variance distance.  ys is the graph of density, and avgs is
##the graph of the average amount of ink.
xs = []
ys = []
avgs = []
for i in range(accuracy+1):
    x = (i*lowpoint)/(accuracy)
    y = f(x, lowpoint)
    xs.append(x)
    ys.append(y)
    avgs.append(avg(lowpoint))

plt.plot(xs, ys, 'b-')
plt.plot(xs, avgs, 'r-')
plt.plot(0, 0, 'w.')
plt.plot(lowpoint/2, 1, 'w.')
plt.show()
\end{verbatim}
Running the code with an accuracy of $100000$ I get $1.69182$ as the value of $a$ that gives the lowest variation, with a variation of roughly $.00737$.  This gives an overlap of $.30818$ cm, and a standard deviation of roughly $.08587$ (again, because of the factor of two, this is actually a standard deviation of $.17174$).  Additionally, I will show the two graphs that this code outputs (the graph of the variation and the graph of ink density; in the graph of ink density, the red line is the average value).

\includegraphics[scale=1.0]{June5Figure1}

\includegraphics[scale=1.0]{June5Figure2}

Taking a look at the second graph, the key point seems to be that you minimize how much the second graph goes above $1$ while also limiting how low it goes.  Plugging in $a = \sqrt{3} \approx 1.7321$, one would get the middle peak to be exactly at $y = 1$, and that isn't too far off of what is actually optimal.
\end{document}