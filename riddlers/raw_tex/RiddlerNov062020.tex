\documentclass[11pt]{article}
\usepackage{mathrsfs}
\usepackage{amssymb}
\usepackage{amsmath}
\usepackage{amsthm}
\usepackage{amscd}
\usepackage{epstopdf}
\usepackage{enumerate}
\usepackage{hyperref}
\usepackage{listings}
\usepackage{graphicx}
\usepackage{tikz}
\usepackage{amstext} % for \text macro
\usepackage{array}   % for \newcolumntype macro
\newcolumntype{L}{>{$}l<{$}} % math-mode version of "l" column type
\usepackage[normalem]{ulem}
\allowdisplaybreaks
\graphicspath{{./Images/}}
\hypersetup{
    colorlinks=true,
    linkcolor=blue,
    filecolor=magenta,      
    urlcolor=cyan,
}

\textwidth = 6.5 in
\textheight = 9 in
\oddsidemargin = 0.0 in
\evensidemargin = 0.0 in
\topmargin = 0.0 in
\headheight = 0.0 in
\headsep = 0.0 in
\parskip = 0.2in
\parindent = 0.0in
\newtheorem{theorem}{Theorem}
\newtheorem{conjecture}{Conjecture}
\newtheorem{claim}[theorem]{Claim}
\newtheorem{question}[theorem]{Question}
\newtheorem{problem}[theorem]{Problem}
\newtheorem{lemma}[theorem]{Lemma}
\newtheorem{proposition}[theorem]{Proposition}
\newtheorem{observation}[theorem]{Observation}
\newtheorem{corollary}[theorem]{Corollary}
\newtheorem{Theorem}{Theorem}[section]
\newtheorem{Claim}[Theorem]{Claim}
\newtheorem{Lemma}[Theorem]{Lemma}
\newtheorem{Proposition}[Theorem]{Proposition}
\newtheorem{Corollary}[Theorem]{Corollary}
\newtheorem{definition}[theorem]{Definition}
\newtheorem{example}{Example}
\newtheorem{assumption}{Assumption}
\newtheorem{aside}{Aside}
\newtheorem{fact}{Fact}


\theoremstyle{definition}
\newtheorem{exercise}[theorem]{Exercise}
\newtheorem{remark}[theorem]{Remark}
\newcommand{\N}{\mathbb{N}}
\newcommand{\Q}{\mathbb{Q}}
\newcommand{\Z}{\mathbb{Z}}
\newcommand{\R}{\mathbb{R}}
\newcommand{\C}{\mathbb{C}}
\newcommand{\F}{\mathbb{F}}
\newcommand{\Hom}{\mathrm{Hom}}

\title{FiveThirtyEight's November 6, 2020 Riddler}
\author{Emma Knight}
\date{\today}

\begin{document}
\maketitle
This week's riddler is about simulating fair coin flips:
\begin{question}
Consider a weighted coin that flips to heads with probability $p$.  Suppose I want to simulate a fair coin in at most three (extra credit: $N$) flips. For which values of $p$ is this possible?
\end{question}
There are eight possible outcomes of three coin flips.  One can simulate a fiar coin by choosing a subset of these eight outcomes and declaring that ``heads'' is when one of these outcomes occurs, and ``tails'' is when one of these outcomes doesn't occur.  Then, it is a fair coin if the probability of ``heads'' is exactly $1/2$.  There is one event with probability $p^3$, three with probability $p^2(1-p)$, three with probability $p(1-p)^2$, and one with probability $(1-p)^3$.  Then, one asks how many solutions are there to $ap^3 + bp^2(1-p) + cp(1-p)^2 + d(1-p)^3$ with $a$ and $d \in \{0, 1\}$ and $b$ and $c \in \{ 0, 1, 2, 3\}$ which lie in the interval $[0, 1]$.  However, flipping the heads and tails sets doesn't change whether the coin is fair, so we can always assume that $d=0$ for finding the valid probabilities.  Below is a table of all values of $a, b,$ and $c$ that have at least one solution in the interval, and what all the solutions are.  There are approximate values for every solution, and exact values for solutions that aren't obnoxious to expand out.

{\renewcommand{\arraystretch}{1.3}
\begin{tabular}{L|L}
a, b, c & \text{solutions} \\
\hline \hline
1, 0, 0 & .7937\ldots = \sqrt[3]{1/2}\\ \hline
1, 1, 0 & .7071\ldots = \sqrt{1/2} \\ \hline
1, 2, 0 & .5969\ldots \\ \hline
1, 3, 0 & .5 \\ \hline
1, 0, 1 & .7718\ldots \\ \hline
1, 1, 1 & .6478\ldots \\ \hline
1, 2, 1 & .5 \\ \hline
0, 3, 1 & .5, .7071\ldots = \sqrt{1/2} \\ \hline
1, 3, 1 & .4030\ldots \\ \hline
1, 0, 2 & .7347\ldots \\ \hline
1, 1, 2 & .5 \\
\end{tabular} \hfill
\begin{tabular}{L|L}
a, b, c & \text{solutions} \\ \hline \hline
0, 2, 2 & .5 \\ \hline
1, 2, 2 & .3522\ldots \\ \hline
0, 3, 2 & .3154\ldots, .7627\ldots \\ \hline
1, 3, 2 & .2928\ldots = 1-\sqrt{1/2} \\ \hline
1, 0, 3 & .5 \\ \hline
0, 1, 3 & .2928\ldots = 1-\sqrt{1/2}, .5 \\ \hline
1, 1, 3 & .2653\ldots \\ \hline
0, 2, 3 & .2372\ldots, .6845\ldots \\ \hline
1, 2, 3 & .2281\ldots \\ \hline
0, 3, 3 & .2113\ldots, .7886\ldots = (3\pm \sqrt{3})/6 \\ \hline
1, 3, 3 & .2063\ldots = 1-\sqrt[3]{1/2}
\end{tabular}
}

In total, there are $19$ solutions: $1$ rational number, $4$ quadratic irrationals, and $14$ cubic irrationals.  All the cubic solutions were distinct (something that isn't true of the rational or quadratic solutions).  Of note, if $p$ gives a solution, then so does $1-p$ by changing the roles of heads and tails in your flips.  These were computed with Wolfram Alpha.

\end{document}