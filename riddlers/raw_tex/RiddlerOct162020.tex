\documentclass[11pt]{article}
\usepackage{mathrsfs}
\usepackage{amssymb}
\usepackage{amsmath}
\usepackage{amsthm}
\usepackage{amscd}
\usepackage{epstopdf}
\usepackage{enumerate}
\usepackage{hyperref}
\usepackage{listings}
\usepackage{graphicx}
\usepackage{tikz}
\usepackage{amstext} % for \text macro
\usepackage{array}   % for \newcolumntype macro
\newcolumntype{L}{>{$}l<{$}} % math-mode version of "l" column type
\usepackage[normalem]{ulem}
\allowdisplaybreaks
\graphicspath{{./Images/}}
\hypersetup{
    colorlinks=true,
    linkcolor=blue,
    filecolor=magenta,      
    urlcolor=cyan,
}

\textwidth = 6.5 in
\textheight = 9 in
\oddsidemargin = 0.0 in
\evensidemargin = 0.0 in
\topmargin = 0.0 in
\headheight = 0.0 in
\headsep = 0.0 in
\parskip = 0.2in
\parindent = 0.0in
\newtheorem{theorem}{Theorem}
\newtheorem{conjecture}{Conjecture}
\newtheorem{claim}[theorem]{Claim}
\newtheorem{question}[theorem]{Question}
\newtheorem{problem}[theorem]{Problem}
\newtheorem{lemma}[theorem]{Lemma}
\newtheorem{proposition}[theorem]{Proposition}
\newtheorem{observation}[theorem]{Observation}
\newtheorem{corollary}[theorem]{Corollary}
\newtheorem{Theorem}{Theorem}[section]
\newtheorem{Claim}[Theorem]{Claim}
\newtheorem{Lemma}[Theorem]{Lemma}
\newtheorem{Proposition}[Theorem]{Proposition}
\newtheorem{Corollary}[Theorem]{Corollary}
\newtheorem{definition}[theorem]{Definition}
\newtheorem{example}{Example}
\newtheorem{assumption}{Assumption}
\newtheorem{aside}{Aside}
\newtheorem{fact}{Fact}


\theoremstyle{definition}
\newtheorem{exercise}[theorem]{Exercise}
\newtheorem{remark}[theorem]{Remark}
\newcommand{\N}{\mathbb{N}}
\newcommand{\Q}{\mathbb{Q}}
\newcommand{\Z}{\mathbb{Z}}
\newcommand{\R}{\mathbb{R}}
\newcommand{\C}{\mathbb{C}}
\newcommand{\F}{\mathbb{F}}
\newcommand{\Hom}{\mathrm{Hom}}

\title{FiveThirtyEight's October 16, 2020 Riddler}
\author{Emma Knight}
\date{\today}

\begin{document}
\maketitle
This week's riddler is about strategizing for The Price is Right:
\begin{question}
This week, we return to the brilliant and ageless game show, “The Price is Right.” In a modified version of the bidding round, you and two (not three) other contestants must guess the price of an item, one at a time.

Assume the true price of this item is a randomly selected value between 0 and 100. (Note: The value is a real number and does not have to be an integer.) Among the three contestants, the winner is whoever guesses the closest price without going over. For example, if the true price is 29 and I guess 30, while another contestant guesses 20, then they would be the winner even though my guess was technically closer.

In the event all three guesses exceed the actual price, the contestant who made the lowest (and therefore closest) guess is declared the winner. I mean, someone has to win, right?

If you are the first to guess, and all contestants play optimally (taking full advantage of the guesses of those who went before them), what are your chances of winning?
\end{question}
The first thing I will do is rescale it so that guesses are between $0$ and $1$ just to make my mental ability to do this easier.  Additionally, $\epsilon$ will denote a number that can be chosen to be as small as desired, and $\epsilon'$ is a number that is as small as desired and smaller than $\epsilon$.

Let's start by figuring out what the optimal strategy is for the third player.  Assume that the two prices named so far are $a$ and $b$.  Player three then has three strategies:
\begin{enumerate}
\item Guess anything between $0$ and $a$.  If they do this, they win if the price is between $0$ and $a$ (as either everyone overshot the price and player three was the lowest, or players one and two overshot the price and player three didn't) and lose otherwise.
\item Guess $a+\epsilon$.  This wins if the price is between $a+\epsilon$ and $b$ and loses otherwise.
\item Guess $b+\epsilon$.  This wins if the price is over $b+\epsilon$ and loses otherwise.
\end{enumerate}
Thus, player three can get all but $\epsilon$ of the longest interval remaining and will do so.  There is an important note about tie-breaking here: if the first interval is as long as either the second or third interval, then player three can get all of the first interval but not all of either the second or third interval and will do so.  However, if the second interval and third interval have the same length, then there is no reason to prefer one over the other (and in particular, the the first two players cannot rely on player three behaving in a predictible fashion).

Now, let's think about this from player two's perspective.  Assume that player one has guessed $x$.  Player two has two strategies availiable to them:
\begin{enumerate}
\item Player two can guess $x/2$.  If they do so, then player three will either guess something in the interval $[0, x/2)$ giving player three an $x/2$ chance of winning and player two an $x/2$ chance of winning, or guess $x + \epsilon$, giving player three an $1-x-\epsilon$ chance of winning and player two an $x$ chance of winning.  Player two can do no better by guessing elsewhere in the interval $[0, x)$ because either player three will guess $x+\epsilon$ leaving the odds of player two winning unchanged or will take (most of) the longer interval, reducing player two's chances of winning.
\item Player two can guess $x + y$.  In this case, player two will win $1-x-y$ of the time if either $x$ or $y$ is larger than $1-x-y$, and will win $\epsilon$ of the time if $1-x-y$ is larger than both $x$ and $y$.  For this to be a sensible strategy, one must have that $x\geq1-x-y$ or $y>1-x-y$ (the difference in inequalities comes from the tiebreak rules above).  Since player two wants to choose as small of a value of $y$ as possible (so as to maximize $1-x-y$), they will either choosethe smallest positive value among $1-2x$, $1/2-x/2+\epsilon$, or $\epsilon$.
\end{enumerate}
Strategy $1$ produces an $x$ chance of winning if $0\leq x < 2/3$ and an $x/2$ chance of winning if $2/3\leq x \leq 1$.  Conversely, strategy $2$ produces a $1/2 - x/2 - \epsilon$ chance of winning if $0 \leq x \leq 1/3$, an $x$ chance of winning if $1/3<x<1/2$, and a $1-x-\epsilon$ chance of winning if $1/2 \leq x \leq 1$.  They will do strategy $2$ when $x$ is in the interval $[0, 1/3)$, can do either strategy if $x \in [1/3, 1/2)$, and will do strategy $1$ if $x \in [1/2, 1]$.

So now, all that is left is to compute player one's chances of winning for each $x$ value.
\begin{enumerate}
\item If $x \in [0, 1/3)$, player two will choose $1/2 + x/2 + \epsilon$ and then player three will choose $x + \epsilon'$, leaving player one with an $x + \epsilon'$ chance of winning.
\item If $x \in [1/3, 1/2)$, player two may (and so player one must assume that they will) choose strategy $1$.  If so, this means player three will choose $x+\epsilon$, giving player one an $\epsilon$ chance of winning.
\item If $x \in [1/2, 2/3)$, player two will choose strategy $1$.  Again, player three will choose $x+ \epsilon$, leaving player with an $\epsilon$ chance of winning.
\item If $x \in [2/3, 1]$, player two will still choose strategy $1$.  However, this time, player three will choose some thing in the interval $[0, x/2)$, leaving player one with a $1-x$ chance of winning.
\end{enumerate}
Thus, if player one chooses $2/3$, then they get a $1/3$ chance of winning as player two chooses $1/3$ and player three chooses some number less than $1/3$.  If they choose anything else, it will work out worse for them, so this is their best strategy.
\end{document}