\documentclass[11pt]{article}
\usepackage{mathrsfs}
\usepackage{amssymb}
\usepackage{amsmath}
\usepackage{amsthm}
\usepackage{amscd}
\usepackage{epstopdf}
\usepackage{enumerate}
\usepackage{hyperref}
\usepackage{listings}
\usepackage{graphicx}
\usepackage{tikz}
\usepackage{amstext} % for \text macro
\usepackage{array}   % for \newcolumntype macro
\newcolumntype{L}{>{$}l<{$}} % math-mode version of "l" column type
\usepackage[normalem]{ulem}
\allowdisplaybreaks
\graphicspath{{./Images/}}
\hypersetup{
    colorlinks=true,
    linkcolor=blue,
    filecolor=magenta,      
    urlcolor=cyan,
}

\textwidth = 6.5 in
\textheight = 9 in
\oddsidemargin = 0.0 in
\evensidemargin = 0.0 in
\topmargin = 0.0 in
\headheight = 0.0 in
\headsep = 0.0 in
\parskip = 0.2in
\parindent = 0.0in
\newtheorem{theorem}{Theorem}
\newtheorem{conjecture}{Conjecture}
\newtheorem{claim}[theorem]{Claim}
\newtheorem{question}[theorem]{Question}
\newtheorem{problem}[theorem]{Problem}
\newtheorem{lemma}[theorem]{Lemma}
\newtheorem{proposition}[theorem]{Proposition}
\newtheorem{observation}[theorem]{Observation}
\newtheorem{corollary}[theorem]{Corollary}
\newtheorem{Theorem}{Theorem}[section]
\newtheorem{Claim}[Theorem]{Claim}
\newtheorem{Lemma}[Theorem]{Lemma}
\newtheorem{Proposition}[Theorem]{Proposition}
\newtheorem{Corollary}[Theorem]{Corollary}
\newtheorem{definition}[theorem]{Definition}
\newtheorem{example}{Example}
\newtheorem{assumption}{Assumption}
\newtheorem{aside}{Aside}
\newtheorem{fact}{Fact}


\theoremstyle{definition}
\newtheorem{exercise}[theorem]{Exercise}
\newtheorem{remark}[theorem]{Remark}
\newcommand{\N}{\mathbb{N}}
\newcommand{\Q}{\mathbb{Q}}
\newcommand{\Z}{\mathbb{Z}}
\newcommand{\R}{\mathbb{R}}
\newcommand{\C}{\mathbb{C}}
\newcommand{\F}{\mathbb{F}}
\newcommand{\Hom}{\mathrm{Hom}}

\title{FiveThirtyEight's September 11, 2020 Riddler}
\author{Emma Knight}
\date{\today}

\begin{document}
\maketitle
Today's riddler is about the Tour de FiveThirtyEight:
\begin{question}
At the moment, you are racing against three other riders up a mountain. The first rider over the top gets $5$ points, the second rider gets $3$, the third rider gets $2$, and the fourth rider gets $1$.

All four of you are of equal ability - that is, under normal circumstances, you all have an equal chance of reaching the summit first. But there’s a catch - two of your competitors are on the same team. Teammates are able to work together, drafting and setting a tempo up the mountain. Whichever teammate happens to be slower on the climb will get a boost from their faster teammate, and the two of them will both reach the summit at the faster teammate’s time.

As a lone rider, the odds may be stacked against you. In your quest for the polka dot jersey, how many points can you expect to win on this mountain, on average?
\end{question}
Let's call the riders $A$, $B$, $C1$, and $C2$, with $A$ being you.  There are $24$ possible finishes for the four racers, but since distinguishing between $C1$ and $C2$ is irrelevant, there are only $12$ relevant orders.  I will list them below together with the number of points you get:
\begin{itemize}
\item $ABCC - 5$
\item $ACBC - 5$
\item $ACCB - 5$
\item $BACC - 3$
\item $CABC - 2$
\item $CACB - 2$
\item $BCAC - 1$
\item $CBAC - 1$
\item $CCAB - 2$
\item $BCCA - 1$
\item $CBCA - 1$
\item $CCBA - 1$
\end{itemize}
One gets a total score of $29$, so the average result is $\displaystyle{\frac{29}{12}}$.  
\end{document}